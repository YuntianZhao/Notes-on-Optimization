% kaobook
% LaTeX Template
% Version 1.3 (December 9, 2021)
%
% This template originates from:
% https://www.LaTeXTemplates.com
%
% For the latest template development version and to make contributions:
% https://github.com/fmarotta/kaobook
%
% Authors:
% Federico Marotta (federicomarotta@mail.com)
% Based on the doctoral thesis of Ken Arroyo Ohori (https://3d.bk.tudelft.nl/ken/en)
% and on the Tufte-LaTeX class.
% Modified for LaTeX Templates by Vel (vel@latextemplates.com)
%
% License:
% CC0 1.0 Universal (see included MANIFEST.md file)

% Notes on Optimal Control
% Yuntian Zhao

%----------------------------------------------------------------------------------------
%	PACKAGES AND OTHER DOCUMENT CONFIGURATIONS
%----------------------------------------------------------------------------------------

\documentclass[
	a4paper, % Page size (e.g., b5paper)
	fontsize=10pt, % Base font size
	twoside=false, % Use different layouts for even and odd pages (in particular, if twoside=true, the margin column will be always on the outside)
	%open=any, % If twoside=true, uncomment this to force new chapters to start on any page, not only on right (odd) pages
	%chapterentrydots=true, % Uncomment to output dots from the chapter name to the page number in the table of contents
	numbers=noenddot, % Comment to output dots after chapter numbers; the most common values for this option are: enddot, noenddot and auto (see the KOMAScript documentation for an in-depth explanation)
]{kaobook}

% Choose the language
\ifxetexorluatex
	\usepackage{polyglossia}
	\setmainlanguage{english}
\else
	\usepackage[english]{babel} % Load characters and hyphenation
\fi
\usepackage[english=american]{csquotes}	% English quotes
% Load packages for testing
\usepackage{blindtext}
% \usepackage{showframe} % Uncomment to show boxes around the text area, margin, header and footer
% \usepackage{showlabels} % Uncomment to output the content of \label commands to the document where they are used

% Load the bibliography package
\usepackage{kaobiblio}
\addbibresource{main.bib} % Bibliography file

% Load mathematical packages for theorems and related environments
\usepackage[framed=false]{kaotheorems}

% Load the package for hyperreferences
\usepackage{kaorefs}

\usepackage{snippets}

\graphicspath{{examples/documentation/images/}{images/}} % Paths in which to look for images

\makeindex[columns=2, title=Alphabetical Index, intoc] % Make LaTeX produce the files required to compile the index

\makeglossaries % Make LaTeX produce the files required to compile the glossary
% \newglossaryentry{computer}{
% 	name=computer,
% 	description={is a programmable machine that receives input, stores and manipulates data, and provides output in a useful format}
% }

% % Glossary entries (used in text with e.g. \acrfull{fpsLabel} or \acrshort{fpsLabel})
% \newacronym[longplural={Frames per Second}]{fpsLabel}{FPS}{Frame per Second}
% \newacronym[longplural={Tables of Contents}]{tocLabel}{TOC}{Table of Contents}

 % Include the glossary definitions

\makenomenclature % Make LaTeX produce the files required to compile the nomenclature

% Reset sidenote counter at chapters
%\counterwithin*{sidenote}{chapter}

%----------------------------------------------------------------------------------------

\begin{document}

%----------------------------------------------------------------------------------------
%	BOOK INFORMATION
%----------------------------------------------------------------------------------------

% \titlehead{The \texttt{kaobook} class}
% \subject{Use this document as a template}

\title[Optimization]{Notes on Optimization}
\subtitle{Static and Dynamic Optimization}

\author[Yuntian Zhao]{Yuntian Zhao\thanks{email: \texttt{1710313@i.pkuschool.edu.cn}}}

\date{\today}

% \publishers{An Awesome Publisher}

%----------------------------------------------------------------------------------------

\frontmatter % Denotes the start of the pre-document content, uses roman numerals

%----------------------------------------------------------------------------------------
%	OPENING PAGE
%----------------------------------------------------------------------------------------

% \makeatletter
% \extratitle{
% 	% In the title page, the title is vspaced by 9.5\baselineskip
% 	\vspace*{9\baselineskip}
% 	\vspace*{\parskip}
% 	\begin{center}
% 		% In the title page, \huge is set after the komafont for title
% 		\usekomafont{title}\huge\@title
% 	\end{center}
% }
% \makeatother

%----------------------------------------------------------------------------------------
%	COPYRIGHT PAGE
%----------------------------------------------------------------------------------------

% \makeatletter
% \uppertitleback{\@titlehead} % Header

% \lowertitleback{
% 	\textbf{Disclaimer}\\
% 	You can edit this page to suit your needs. For instance, here we have a no copyright statement, a colophon and some other information. This page is based on the corresponding page of Ken Arroyo Ohori's thesis, with minimal changes.
	
% 	\medskip
	
% 	\textbf{No copyright}\\
% 	\cczero\ This book is released into the public domain using the CC0 code. To the extent possible under law, I waive all copyright and related or neighbouring rights to this work.
	
% 	To view a copy of the CC0 code, visit: \\\url{http://creativecommons.org/publicdomain/zero/1.0/}
	
% 	\medskip
	
% 	\textbf{Colophon} \\
% 	This document was typeset with the help of \href{https://sourceforge.net/projects/koma-script/}{\KOMAScript} and \href{https://www.latex-project.org/}{\LaTeX} using the \href{https://github.com/fmarotta/kaobook/}{kaobook} class.
	
% 	The source code of this book is available at:\\\url{https://github.com/fmarotta/kaobook}
	
% 	(You are welcome to contribute!)
	
% 	\medskip
	
% 	\textbf{Publisher} \\
% 	First printed in May 2019 by \@publishers
% }
% \makeatother

%----------------------------------------------------------------------------------------
%	DEDICATION
%----------------------------------------------------------------------------------------

% \dedication{
% 	The harmony of the world is made manifest in Form and Number, and the heart and soul and all the poetry of Natural Philosophy are embodied in the concept of mathematical beauty.\\
% 	\flushright -- D'Arcy Wentworth Thompson
% }

%----------------------------------------------------------------------------------------
%	OUTPUT TITLE PAGE AND PREVIOUS
%----------------------------------------------------------------------------------------

% Note that \maketitle outputs the pages before here

\maketitle

%----------------------------------------------------------------------------------------
%	PREFACE
%----------------------------------------------------------------------------------------

\chapter*{Preface}
\addcontentsline{toc}{chapter}{Preface} % Add the preface to the table of contents as a chapter

Nothing to say here.

\begin{flushright}
\textit{Yuntian Zhao}
\end{flushright}

\index{preface}

%----------------------------------------------------------------------------------------
%	TABLE OF CONTENTS & LIST OF FIGURES/TABLES
%----------------------------------------------------------------------------------------

\begingroup % Local scope for the following commands

% Define the style for the TOC, LOF, and LOT
%\setstretch{1} % Uncomment to modify line spacing in the ToC
%\hypersetup{linkcolor=blue} % Uncomment to set the colour of links in the ToC
\setlength{\textheight}{230\hscale} % Manually adjust the height of the ToC pages

% Turn on compatibility mode for the etoc package
\etocstandarddisplaystyle % "toc display" as if etoc was not loaded
\etocstandardlines % "toc lines" as if etoc was not loaded

\tableofcontents % Output the table of contents

\listoffigures % Output the list of figures

% Comment both of the following lines to have the LOF and the LOT on different pages
\let\cleardoublepage\bigskip
\let\clearpage\bigskip

\listoftables % Output the list of tables

\endgroup

%----------------------------------------------------------------------------------------
%	MAIN BODY
%----------------------------------------------------------------------------------------

\mainmatter % Denotes the start of the main document content, resets page numbering and uses arabic numbers
\setchapterstyle{kao} % Choose the default chapter heading style

% \input{chapters/introduction.tex}

\pagelayout{wide} % No margins
\addpart{Static Optimization}
\pagelayout{margin} % Restore margins

\setchapterstyle{kao}
\setchapterpreamble[u]{\margintoc}
\chapter{Variations}
\labch{variations}

\section{Unconstrained minimizer}

\begin{definition}[Unconstrained optimization]
\labdef{unconstrained_optimization}
\index{unconstrained optimization}
A point $x_0 \in \R^n$ is a minimizer of the function $f: \R^n \to \R$ if $f(x0) \le f(x)$ for all $x \in \R^n$.
And we write 
\begin{align}
f(x_0) = \min_{x\in\R^n} f(x).
\end{align}
\end{definition}

\subsection{First variation}

\begin{theorem}[First variation]
Assume that $f: \R^n \to \R$ is differentiable and $x_0$ is a minimizer.
Then
\begin{align}
\grad f(x_0) = 0.
\end{align}
Equivalently, we have
\begin{align}
\frac{\pp f}{\pp x_i} f(x_0) = 0, \quad i = 1, \dots, n.
\end{align}
\end{theorem}

\begin{proof}
Let us define $\phi(t) = f(x_0 + ty)$.
Since $x_0$ is a minimizer,
\begin{align}
\phi(0) = f(x_0) \le f(x_0 + ty) = \phi(t)
\end{align}
we apply the chain rule from multivariable calculus:
\begin{align}
\phi'(t) = \sum _{i=1}^{n} \frac{\pp f}{\pp x_i} f(x_0 + ty) y_i.
\end{align}
For $t=0$ and all $y \in \R^n$,
\begin{align}
0 = \phi'(0) &= \sum _{i=1}^{n} \frac{\pp f}{\pp x_i} f(x_0) y_i = \grad f(x_0) ^T y.
\end{align}
Finally, by taking $y=\grad f(x_0)$, we have $\grad f(x_0) = 0$.
\end{proof}

\begin{definition}[Critical point, extremal]
\index{cirtical point}
\index{extremal}
A point $x \in \R^n$ is called a critical point (or an extremal) of $f$ if $\grad f(x)=0$.
\end{definition}

\begin{remark}
For a differentiable function $f$, if $x_0$ is a minimizer, then it is a critical point.
\end{remark}

\subsection{Second variation}

\begin{definition}
\labdef{nonnegative_definite}
A symmetric $n \times n$ matrix $A$ is nonnegative definite if for all $y \in \R^n$, 
\begin{align}
y^T A y = \sum_{i=1}^{n} \sum_{j=1}^{n} a_{ij}y_{i}y_{j} \ge 0.
\end{align}
And in this case, we write
\begin{align}
A \succeq 0.
\end{align}
\end{definition}

\begin{theorem}[Second variation]
Assume $f: \R^n \to \R$ is twice differentiable and $x_0$ is a minimizer.
Then
\begin{align}
\grad^2 f(x_0) \succeq 0.
\end{align}
\end{theorem}

\begin{proof}
Define $\phi(t) = f(x_0 + ty)$.
we again apply the chain rule from multivariable calculus:
\begin{align}
\phi''(t) = \sum_{i=1}^{n}\sum_{j=1}^{n} \frac{\pp ^2 f}{\pp x_i \pp x_j} (x_0 + t y) y_i y_j.
\end{align}
Since $t = 0$ is a minimizer of $\phi$ which suggests $\phi''(0) \ge 0$, we have
\begin{align}
\phi''(0) = \sum_{i=1}^{n}\sum_{j=1}^{n} \frac{\pp ^2 f}{\pp x_i \pp x_j} (x_0) y_i y_j \ge 0.
\end{align}
By \refdef{nonnegative_definite}, $\grad^2 f(x_0) \succeq 0$.
\end{proof}

\section{Equality constraints}

We turn our attention now to optimization problems with constraints.
We assume $f, g_1, \dots, g_m : \R^n \to \R$ are twice continuously differentiable functions.
Then we define $g:\R^n \to \R^m$ by

\begin{align}
g = \begin{bmatrix}
g_1\\
\vdots\\
g_m\\
\end{bmatrix}.
\end{align}

Then the gradient of $g$ is

\begin{align}
\grad g = \begin{bmatrix}
(\grad g_1)^T\\
\vdots\\
(\grad g_m)^T\\
\end{bmatrix}
= \begin{bmatrix}
\frac{\pp g_1}{\pp x_1} & \cdots & \frac{\pp g_1}{\pp x_n} \\
\vdots & \ddots & \vdots \\
\frac{\pp g_m}{\pp x_1} & \cdots & \frac{\pp g_m}{\pp x_n} \\
\end{bmatrix}.
\end{align}
This is an $m \times n$ matrix-valued function.

\begin{definition}[Constrained optimization]
\labdef{constrained_optimization}
A constrained optimization problem is to find $x_0 \in \R^n$ that
\begin{align}
\min_{x\in\R^n} \ & f(x) \\
\st & g(x) = 0.
\end{align}
\end{definition}

\begin{definition}
A point $x \in \R^n$ is called feasible for if $g(x) = 0$; that is, if $g_k(x) = 0$ for $k = 1, \dots, m$.
\end{definition}

\subsection{Lagrange multipliers}

The method of Lagrange multipliers reveals a linear relationship between the gradients of $f, g1, \dots, gm$ at a minimizer $x_0$.

\subsection{Constrained first variation}

\begin{theorem}[F. John's form of the constrained first variation formula]
\labthm{FJ_constrained_first_variation_formula}
Suppose $x_0$ solves the constrained optimization problem defined in \refdef{constrained_optimization}.
Then there exist real numbers $\gamma_0, \lambda_0^1, \dots, \lambda_0^m$ (not all equal to 0) such that
\begin{align}
\gamma_0 \grad f(x_0) + \sum_{k=1}^{n} \lambda_0^k \grad g_k(x_0) = 0.
\end{align}
By defining $\lambda_0 = [ \lambda_0^1 \dots \lambda_0^m ]^T$, equivalently we have
\begin{align}
\gamma_0 \grad f(x_0) + \lambda_0^T \grad g(x_0) = 0. \labeq{FJcfvf_mat}
\end{align}
\end{theorem}

\begin{remark}
When $\gamma_0 = 0$, we call it an abnormal multiplier.
If $\gamma_0 \neq 0$, it is a normal multiplier, in which case we can divide and convert to the case $\gamma_0 = 1$ (for possibly new constants $\lambda_0^1, \dots, \lambda_0^m$.)
\end{remark}

\begin{proof}
Step 1: Fix $\beta >0$.
For each $\alpha>0$, define
\begin{align}
F^{\alpha}(x) = f(x) + \frac{\alpha}{2} |g(x)|^2 + \frac{\beta}{2} |x-x_0|^2
\labeq{proof_FJcfvf_eq_1}
\end{align}
We will later send $\alpha \to \infty$; this procedure is the penalty method.
Let $B=\{ x \in \R^n : |x-x_0| \le 1 \}$. 
The Extreme Value Theorem tells us that for continuous $F^\alpha: B\to\R$ and closed and bounded $B$, there exists $x_\alpha$ that minimizes $F^\alpha$.
Then we have $F^\alpha(x_\alpha) = \min _{x \in B} F^\alpha(x)$.
\begin{align}
f(x_\alpha) + \frac{\alpha}{2} |g(x_\alpha)|^2 + \frac{\beta}{2} |x_\alpha-x_0|^2 = F^{\alpha}(x_\alpha) \le F^{\alpha}(x_0) = f(x_0),
\labeq{proof_FJcfvf_eq_2}
\end{align}
since $g(x_0) = 0$.
Because $\alpha |g(x_\alpha)|^2 \le 2 (f(x_0) - f(x_\alpha)) - \beta |x_\alpha-x_0|^2$, $\{ \alpha |g(x_\alpha)|^2 \} _{\alpha > 0}$ is bounded.
Consequently,
\begin{align}
\lim _{\alpha \to \infty} g(x_\alpha) = 0.
\labeq{proof_FJcfvf_eq_3}
\end{align}

Step 2: Next, we use the Bolzano-Weierstrass Theorem to select a convergent subsequence $\{ x_{\alpha_j} \}_{j=1}^{\infty}$ from $\{ x_{\alpha} \} \subset B$ so that $x_{\alpha_j} \to \bar{x}$ as $\alpha_j \to \infty$, for some $\bar{x} \in B$.
\refeq{proof_FJcfvf_eq_3} shows that $g(\bar{x})=0$ and therefore $\bar{x}$ is feasible.
Hence $f(x_0) \le f(\bar{x})$
Moreover, \refeq{proof_FJcfvf_eq_2} gives 
\begin{align}
f(\bar{x}) + \frac{\beta}{2} |\bar{x}-x_0|^2 \le f(x_0) \le f(\bar{x}),
\labeq{proof_FJcfvf_eq_4}
\end{align}
which means that $|\bar{x}-x_0|^2 = 0$.
So, $\bar{x} = x_0$.
This is true for all convergent subsequences $x_{\alpha_j} \to \bar{x}$ and consequently 
\begin{align}
\lim_{\alpha \to \infty} x_{\alpha} = x_0.
\labeq{proof_FJcfvf_eq_5}
\end{align}

Step 3: So if $\alpha$ is large enough, $F^\alpha$ has a minimum over $B$ at point $x_\alpha$.
And in view of \refeq{proof_FJcfvf_eq_5}, $x_\alpha$ does not lie on the boundary of $B$.
It follows that\sidenote{
We have
\begin{align*}
\grad \left( \frac{|g(x)|^2}{2} \right) 
&= \grad \left( \frac{1}{2} \sum_{i} g_i(x)^2 \right) \\
&= \begin{bmatrix}
\frac{1}{2} \frac{\pp}{\pp x_1} \sum_{i} g_i (x)^2 \\
\vdots \\
\frac{1}{2} \frac{\pp}{\pp x_n} \sum_{i} g_i (x)^2 \\
\end{bmatrix} \\
&= \begin{bmatrix}
\sum_{i} g_i (x) \frac{\pp g_i (x)}{\pp x_1} \\
\vdots \\
\sum_{i} g_i (x) \frac{\pp g_i (x)}{\pp x_n} \\
\end{bmatrix} \\
&= \grad g(x)^T g(x).
\end{align*}
}
\begin{align}
0 = \grad F^\alpha (x_\alpha) = \grad f(x_\alpha) + \alpha \grad g(x_\alpha)^T \grad g(x_\alpha) + \beta (x_\alpha - x_0).
\labeq{proof_FJcfvf_eq_6}
\end{align}
Then
\begin{align}
0 = \gamma_\alpha \grad f(x_\alpha) + \grad g(x_\alpha)^T \lambda_\alpha + \gamma_\alpha \beta (x_\alpha - x_0),
\labeq{proof_FJcfvf_eq_7}
\end{align}
for
\begin{align}
\gamma_\alpha  &= (1+\alpha^2|g(x_\alpha)|^2)^{1/2} \\
\lambda_\alpha &= \gamma_\alpha \alpha g(x_\alpha)
\end{align}

Step 4: Now $\gamma_\alpha^2 + \lambda_\alpha^2 = 1$ and therefore $\{(\gamma_\alpha, \lambda_\alpha)\}_{\alpha >0}$ is bounded.
Consequently, the Bolzano-Weierstrass Theorem asserts that there is a sequence $\alpha_j \to \infty$ such that $
\gamma_{\alpha_j} \to \gamma_0 \in \R$, $\lambda_{\alpha_j} \to \lambda_0 \in \R^m$, and $(\gamma_0, \lambda_0) \neq (0, 0)$.
Let $\alpha$ in \refeq{proof_FJcfvf_eq_7} behaves like $\alpha_j \to \infty$ and recall \refeq{proof_FJcfvf_eq_5} to derive \refeq{FJcfvf_mat}.
We finish the proof.
\end{proof}

\begin{remark}
The term $\beta$ played no role, but will be used in the later proof of \refthm{constrained_second_variation_formula}.

The existence theory for Lagrange multipliers for inequality constraints of the form $h_j (x) \le 0$ for $j=1,\dots, p$ is more complicated and will be discussed later. 
\end{remark}

\subsection{Regular points}

\begin{definition}[Regular point]
\index{regular point}
A point $x_0$ is regular if the vectors $\{\grad g_k(x_0) \} _{k=1,\dots,m}$ are linearly independent in $\R^n$.
\end{definition}

\begin{theorem}[Constrained first variation formula]
\labthm{constrained_first_variation_formula}
Suppose that $x_0$ solves the constrained optimization problem defined in \refdef{constrained_optimization} and furthermore that $x_0$ is regular. Then there exist real numbers $\lambda_0^1,\dots,\lambda_0^m$ such that
\begin{align}
\grad f(x_0) + \sum_{k=1}^{n} \lambda_0^k \grad g_k(x_0) = 0.
\end{align}
By defining $\lambda_0 = [ \lambda_0^1 \dots \lambda_0^m ]^T$, equivalently we have
\begin{align}
\grad f(x_0) + \lambda_0^T \grad g(x_0) = 0. \labeq{cfvf_mat}
\end{align}
\end{theorem}
\begin{proof}
From \refthm{FJ_constrained_first_variation_formula}, we know that 
\begin{align}
\gamma_0 \grad f(x_0) + \lambda_0^T \grad g(x_0) = 0 \labeq{}
\end{align}
for constants $\gamma_0, \lambda_0^1, \dots, \lambda_0^m$ are not all zero.

We claim that if $x_0$ is regular, then $\gamma_0 \neq 0$.
To see this, suppose instead that $\gamma_0 = 0$, then
\begin{align}
\lambda_0^T \grad g(x_0) = 0
\end{align}
and $\lambda_0 \neq 0$. But this is impossible as $\grad g(x_0)$ are independent. 
Thus we divide both sides of \refeq{FJcfvf_mat} by $\gamma_0$ to obtain an expression of the form in \refeq{cfvf_mat}.
\end{proof}

\subsection{Constrained second variation}

We discuss next how to compute second variations when we have regular
equality constraints.

\begin{lemma}
If $x_0$ is regular, then the $m \times m$ matrix $\grad g(x_0) \grad g(x_0)^T$ is nonsingular.
\end{lemma}

\begin{proof}
Suppose that $y \in \R^m$ and $\grad g(x_0) \grad g(x_0)^T y = 0$.
If a nonzero $y$ satisfies this, then $\grad g(x_0) \grad g(x_0)^T$ is singular.
However, if not (only $y=0$ satisfies this), then matrix $\grad g(x_0) \grad g(x_0)^T$ is nonsingular.

We multiply both sides by $y^T$.
\begin{align}
0 = y^T \grad g(x_0) \grad g(x_0)^T y = |\grad g(x_0)^T y|^2
\end{align}
Therefore,
\begin{align}
0 = \grad g(x_0)^T y = 
\begin{bmatrix}
\frac{\pp g_1}{\pp x_1} & \cdots & \frac{\pp g_m}{\pp x_1} \\
\vdots & \ddots & \vdots \\
\frac{\pp g_1}{\pp x_n} & \cdots & \frac{\pp g_m}{\pp x_n} \\
\end{bmatrix}
= \sum _{k=1}^{m} y_k \grad g_k(x_0)
\end{align}
Since $x_0$ is regular, the vectors $\{ \grad g_k(x_0) \} _{k=1,\dots,m}$ are linearly independent in $\R^n$ so that we must have $y=0$.
The $m \times m$ matrix $\grad g(x_0) \grad g(x_0)^T$ is nonsingular.
\end{proof}

\begin{theorem}[Constrained second variation formula]
\labthm{constrained_second_variation_formula}
Suppose that x0 solves the constrained optimization problem defined in \refdef{constrained_optimization} and that $x_0$ is regular.
Let $\lambda_0^1,\dots,\lambda_0^m$ be corresponding Lagrange multipliers, satisfying the first variation formula, \refeq{cfvf_mat}.
Then
\begin{align}
y^T \left(\grad ^2 f(x_0) + \sum_{k=1}^{m}\lambda_0^k \grad ^2 g_k(x_0) \right) y \ge 0 \labeq{csvf}
\end{align}
for all $y \in \R^n$ such that $\grad g(x_0) y = 0$.
\end{theorem}

\begin{proof}
Step 1: We return to the proof of \refthm{FJ_constrained_first_variation_formula} and extract more detailed information. 
Since $x_0$ is regular, we know from \refthm{constrained_first_variation_formula} that
\begin{align}
\gamma_\alpha = (1+\alpha^2|g(x_\alpha)|^2)^{1/2} \to \gamma_0 \neq 0.
\labeq{proof_csvf_eq_1}
\end{align}
Hence $\{ \alpha |g(x_\alpha)| \}_{\alpha > 0}$ is bounded. 
We can assume that $\alpha_j g(x_{\alpha_j}) \to \lambda_0$.

Step 2: Since, $x_\alpha$ lies within the interior of ball $B$ for large $\alpha$, we have $z^T \grad ^2 F^\alpha (x_\alpha) z \ge 0$ for all $z \in \R^n$ where
\begin{align}
\grad ^2 F^\alpha (x) = \grad^2 f(x) + \alpha \grad g(x)^T \grad g(x) + \alpha \sum _{k=1} ^{m} g_k (x) \grad^2 g_k(x) + \beta I.
\labeq{proof_csvf_eq_2}
\end{align}
Now, 
\begin{align}
z^T \left( \grad^2 f(x_\alpha) + \alpha \grad g(x_\alpha)^T \grad g(x_\alpha) + \alpha \sum _{k=1} ^{m} g_k (x_\alpha) \grad^2 g_k(x_\alpha) + \beta I \right) z \ge 0
\labeq{proof_csvf_eq_3}
\end{align}
Remember from the Lemma that $\grad g(x_0) \grad g(x_0)^T$ is nonsingular; consequently $\grad g(x_\alpha) \grad g(x_\alpha)^T$ is invertible for a large $\alpha$.
Given $y \in \R^m$ with $\grad g(x_0) y = 0$, we define
\begin{align}
z_\alpha = y - \grad g(x_\alpha)^T \left( \grad g(x_\alpha) \grad g(x_\alpha)^T \right) ^{-1} \grad g(x_\alpha) y.
\labeq{proof_csvf_eq_4}
\end{align}
Then $\grad g(x_\alpha) y = 0$.
Observe that $z_\alpha \to y$ as $\alpha \to \infty$.
This follows since $x_\alpha \to x_0$ and $\grad g(x_0) y = 0$.

Step 3: Take $z = z_\alpha$ in \refeq{proof_csvf_eq_3}, then
\begin{align}
(z_\alpha)^T \left( \grad^2 f(x_\alpha) + \alpha \grad g(x_\alpha)^T \grad g(x_\alpha) + \alpha \sum _{k=1} ^{m} g_k (x_\alpha) \grad^2 g_k(x_\alpha) + \beta I \right) (z_\alpha) \ge 0,
\labeq{proof_csvf_eq_5}
\end{align}
where $(z_\alpha)^T \alpha \grad g(x_\alpha)^T \grad g(x_\alpha) (z_\alpha) = 0$.
Let $\alpha \to \infty$ and recall that $\alpha_j g(x_{\alpha_j}) \to \lambda_0$,
\begin{align}
y^T \left( \grad ^2 F^\alpha (x_\alpha) = \grad^2 f(x_\alpha) + \sum _{k=1} ^{m} \lambda_0^k g_k (x_\alpha) \grad^2 g_k(x_\alpha) + \beta I \right) y \ge 0,
\end{align}
for $y \in \R^m$ that $\grad g(x_0) y = 0$.
To conclude send $\beta \to 0$.
\end{proof}

\section{Applications}

\subsection{Least squares}

Let $A$ denote an $m \times n$ matrix and assume $b \in \R^m$ is given.
If the linear system $Ax = b$ has no solution, we can nevertheless build an approximate solutions by finding $x_0 \in \R^n$ that solves the minimization problem
\begin{align}
\min _{x \in \R^n} |Ax-b|^2 \labeq{least_squares}
\end{align}

\begin{theorem}
If $A^T A$ is invertible, then the solution of \refeq{least_squares} is
\begin{align}
x_0 = (A^T A)^{-1} A^T b
\end{align}
\end{theorem}

\begin{proof}
Step 1:
Let $f(x) = |Ax-b|^2 = |Ax|^2 - 2 (Ax)^T b + |b|^2$.
We have
\begin{align}
\grad f(x) = 2 A^T A x - 2 A^T b.
\end{align}
So that $\grad f(x_0) = 0$ implies $x_0 = (A^T A)^{-1} A^T b$.

Step 2:
For uniqueness, we must show that $f(x_0) \le f(x_0 + y)$ for all $y \ne 0$.
\begin{align}
f(x_0+y)-f(x_0) &= |A(x_0+y) -b|^2 - |Ax_0-b|^2 \\
&= 2 (Ax_0-b)^TAy+|Ay|^2 \nonumber \\
&= |Ay|^2 \ge 0. \nonumber
\end{align}
\end{proof}

\subsection{Roots of polynomials}

A novel application of Lagrange multipliers by \sidecite{theo_de_jong_lagrange_2009} shows that the existence of a root for a complex polynomial of degree $n \ge 1$:
\begin{align}
f(z) = z^n + a_{n-1} z^{n-1} + \cdots + a_1 z + a_0
\end{align}
The coefficients $a_0, \dots, a_{n-1}$ here are complex numbers.

If we substitute $z = x + iy$, then we can write $f$ in terms of real and imaginary parts:
\begin{align}
f(z) = u(x,y) + i v(x,y),
\end{align}
where $u, v : \R^2 \to \R$ are polynomials.

\begin{lemma}
The functions $u, v$ solve the Cauchy-Riemann equations
\begin{align}
\frac{\pp u}{\pp x} &= \frac{\pp v}{\pp y} \\
\frac{\pp u}{\pp y} &= -\frac{\pp v}{\pp x}
\end{align}
\end{lemma}

\begin{proof}
We first apply induction to $f_n(z) = z^n$.
When $n=1$, $f_1(z) = z = x + iy = u_1(x,y) + iv_1(x,y)$
It is trivial that $u_1 = x, v_1 = y$ solve the Cauchy-Riemann equations.

Then suppose that for $n=k-1$, $u_{k-1}, v_{k-1}$ solve the Cauchy-Riemann equations.

For $n = k$, $f_k(z) = z f_{k-1}(z) = (x+iy)(u_{k-1}+iv_{k-1})$, so that $u_k = xu_{k-1} - yv_{k-1}, v_k = yu_{k-1} + xv_{k-1}$.
And it is easy to find 
\begin{align}
\frac{\pp u_k}{\pp x} &= u_{k-1} + x \frac{\pp u_{k-1}}{\pp x} - y \frac{\pp v_{k-1}}{\pp x} \\
\frac{\pp v_k}{\pp y} &= x \frac{\pp v_{k-1}}{\pp y} + u_{k-1} + y \frac{\pp u_{k-1}}{\pp y} \\
\frac{\pp u_k}{\pp y} &= x \frac{\pp u_{k-1}}{\pp y} - v_{k-1} - y \frac{\pp v_{k-1}}{\pp y} \\
\frac{\pp v_k}{\pp x} &= v_{k-1} + x \frac{\pp v_{k-1}}{\pp x} + y \frac{\pp u_{k-1}}{\pp x}.
\end{align}
Since $u_{k-1}, v_{k-1}$ solve the Cauchy-Riemann equations, $u_{k}, v_{k}$ also solve the Cauchy-Riemann equations.

Moreover, constant terms go into $u,v$ do not matter.
The proof for a general general polynomials follows by linearity.
When $f(z)=a_0 + \sum_{k \ge 1} a_k f_k(z)$, we have $u= a_0 + \sum_{k \ge 1} a_k u_k$ and $v = \sum_{k \ge 1} a_k v_k$.
The functions $u, v$ solve the Cauchy-Riemann equations.
\end{proof}

\begin{theorem}[Fundamental Theorem of Algebra]
There exists a point $z_0 \in \C$ which $f(z_0) = 0$.
\end{theorem}

\begin{proof}
Step 1: 
We introduce the level sets $L_c = \{ (x,y) \in \R^2 : u(x,y)=c \}$ and $M_c = \{ (x,y) \in \R^2 : v(x,y)=c \}$, where $c \in \R$.
Since $u(x, 0) = x^n+ \text{lower order terms}$, the function $u(x, 0)$ takes on infinitely many values.
It follows that the sets $L_c$ are nonempty for infinitely many values of the parameter $c$.
We observe next that except for finitely many values of $c$, we have $\grad u \neq 0$ on $L_c$ and $\grad v \neq 0$ on $M_c$.
To see this, note that $f'$ is a polynomial, and consequently has at most finitely many zeros.
But $f' = \frac{\pp u}{\pp x} + i\frac{\pp v}{\pp x} = \frac{\pp u}{\pp x} - i\frac{\pp u}{\pp y}$\sidenote{The derivatives for complex variable functions.}, according to Cauchy-Riemann equations, and therefore $\grad u \neq 0$ except at finitely many
points.

Step 2:
Select a value of the parameter $c$ so that $L_c \neq \emptyset$ and $\grad u \neq 0$ on $L_c$.
We introduce the constrained optimization problem
\begin{align}
\min \ & v^2 \\
\st  & u = c.
\end{align}
Since $u^2 + v^2 = |f|^2 \to \infty$ as $|z| \to \infty$, we can use the Extreme Value Theorem to show that there exists a point $(x_c, y_c) \in L_c$ solving the constrained optimization problem.
We claim that $v(x_c,y_c) = 0$.
To see this, note that since $\grad u \neq 0$ on $L_c$, $(x_c, y_c)$ is a regular point. 
Hence \refthm{constrained_first_variation_formula} asserts that there exists a Lagrange multiplier $\lambda$ such that $2v(x_c,y_c) \grad v(x_c,y_c) + \lambda \grad u(x_c,y_c) = 0$.
But the Cauchy-Riemann equations imply $\grad u \cdot \grad v = 0$ and $|\grad u| = |\grad v|$.
Since $\grad u(x_c, y_c) \neq 0$, it follows that $\lambda = 0$ and $\grad v(x_c, y_c) = 0$.

Step 3:
In view of $v(x_c,y_c)=0$, we see that $M_0 \neq \emptyset$.
Select a sequence $c_k \to 0$, such that $M_{c_k} \neq \emptyset$ and $\grad v \neq 0$ on $M_{c_k}$.
Then the argument above (with the roles of $u$ and $v$ reversed) shows that
there exist points $(x_k, y_k) \in M_{c_k}$ for which $u(x_k, y_k) = 0$.
The sequence $\{ (x_k, y_k) \}_{k=1}^{\infty}$ is bounded, and so, passing if necessary to a subsequence, we may assume $(x_k, y_k) \to (x_0, y_0) \in M_0$.
Then $u(x_0, y_0) = v(x_0, y_0) = 0$, and therefore $f(z_0) = 0$ for $z_0 = x_0+iy_0$.
\end{proof}
\setchapterstyle{kao}
\setchapterpreamble[u]{\margintoc}
\chapter{Linear Optimization}
\labch{linear_optimization}

Linear optimization theory, most commonly known as linear programming, concerns the minimization of linear functions, subject to affine equality and inequality constraints.

\section{Theory}

\subsection{Basic concepts}

Before start, we introduce some notations.
If $x = [x_1 \dots x_n]^T \in \R^n$, we write $x \ge 0$ to mean that $x_i \ge 0$ for all $i = 1, \dots, n$.
Similarly, $x > 0$ means that $x_i > 0$ for all $i = 1,\dots, n$.

We have $x \ge y$ if $x_i \ge y_i$ for all $i = 1, \dots, n$ and $x > y$ if $x_i > y_i$ for all $i = 1, \dots, n$.

\begin{definition}[Canonical primal linear programming problem]
\labdef{canonical_primal_LP}
Let $c \in \R^n$, $b \in \R^m$ and assume $A$ is an $m \times n$ matrix.
The canonical primal linear programming problem is to find $x_0 \in \R^n$ to
\begin{align}
\min \ & c^T x \tag{$P$} \labeq{LP_P} \\
\st & Ax = b \nonumber \\
& x \ge 0 \nonumber
\end{align}
\end{definition}

\begin{definition}
We say $x \in \R^n$ is feasible if $Ax = b$, $x \ge 0$, that is, if $x$ satisfies the constraints in \arefeq{LP_P}.
We will often call a feasible $x$ a feasible solution.
\end{definition}

\begin{definition}[Canonical dual linear programming problem]
\labdef{canonical_dual_LP}
A canonical dual problem is to find $y_0 \in \R^m$ to
\begin{align}
\max \ & b^T y \tag{$D$} \labeq{LP_D} \\
\st & A^T y \le c \nonumber
\end{align}
\end{definition}

\begin{definition}
We say $y \in \R^m$ is feasible for \arefeq{LP_D} if $A^T y\le c$.
\end{definition}

The most important fact of linear programming is that the primal and
dual problems contain information about each other.

\begin{theorem}[Duality and optimality]
\labthm{d_o}
If $x$ is feasible for \arefeq{LP_P} and $y$ is feasible for \arefeq{LP_D}, then 
\begin{align}
b^T y \le c^T x. \labeq{d_o}
\end{align}

If $x_0$ is feasible for \arefeq{LP_P} and $y_0$ is feasible for \arefeq{LP_D}, and if $b^T y_0 = c^T x_0$ then $x_0$ solves \arefeq{LP_P} and $y_0$ solves \arefeq{LP_D}.
\end{theorem}

\begin{proof}
Let $x, y$ be feasible, $Ax=b, x\ge0$ and $A^T y\le c$.
We have 
\begin{align}
b^T y = (Ax)^T y = x^T A^T y \le x^T c = c^T x. \labeq{proof_d_o}
\end{align}
Suppose $x_0$, $y_0$ are feasible and $b^T y_0 = c^T x_0$.
By \arefeq{proof_d_o}, $b^T y0 \le c^T x$ for all feasible $x$ for \arefeq{LP_P}.
So $c^T x0 \le c^T x$ for all feasible $x$, and thus $x_0$ is optimal for \arefeq{LP_P}.
A similar argument works for $y_0$. 
\end{proof}

Then we introduce some other forms of linear programming problems.

\begin{definition}[Standard linear programming problem]
\labdef{standard_LP}
The standard linear programming problem is to find $x_0 \in \R^n$ to
\begin{align}
\min \ & c^T x \tag{$P^*$} \labeq{LP_Ps} \\
\st & Ax \ge b \nonumber \\
    & x \ge 0 \nonumber
\end{align}
\end{definition}

\begin{definition}[Dual standard linear programming problem]
\labdef{dual_standard_LP}
The dual standard linear programming problem is to find $y_0 \in \R^m$ to
\begin{align}
\max \ & b^T y \tag{$D^*$} \labeq{LP_Ds} \\
\st & A^T y \le c \nonumber \\
    & y \ge 0 \nonumber
\end{align}
\end{definition}

\begin{remark}
Note carefully that we now have an inequality $Ax \ge b$ and an additional sign constraint $y \ge 0$.
\end{remark}

\begin{remark}
\labremark{d_o_standard}
We could check that the duality and optimality theorem (\refthm{d_o}) applies to the standard forms \arefeq{LP_Ps} and \arefeq{LP_Ds}.
\end{remark}

\begin{definition}[General linear programming problem]
\labdef{general_LP}
The general linear programming problem is to find $x_0 \in \R^n$ to
\begin{align}
\min \ & c^T x & \tag{$P^o$} \labeq{LP_Po} \\
\st    & \sum_{j=1}^{n} a_{ij} x_j \ge b_i, & i \in I_1 \nonumber \\
       & \sum_{j=1}^{n} a_{ij} x_j = b_i, & i \in I_2 \nonumber \\
       & x_j \ge 0, & j \in J_1 \nonumber
\end{align}
where $I_1 \cup I_2 = I$, $I_1 \cap I_2 = \emptyset$, $J_1 \subseteq J$, $I = \{ 1, \dots, m \}$, and $J=\{ 1, \dots, n \}$.
Here $I_1$ and $J_1$ are the indices of the inequality constraints.
\end{definition}

\begin{definition}[Dual general linear programming problem]
\labdef{dual_general_LP}
The dual general linear programming problem is to find $y_0 \in \R^m$ to
\begin{align}
\max \ & b^T y & \tag{$D^o$} \labeq{LP_Do} \\
\st & \sum_{i=1}^{m} y_i a_{ij} \le c_j, & j \in J_1 \nonumber \\
    & \sum_{i=1}^{m} y_i a_{ij} = c_j, & j \in J_2 \nonumber \\
    & y_i \ge 0, & i \in I_1 \nonumber
\end{align}
where $J_1 \cup J_2 = J$, $J_1 \cap J_2 = \emptyset$, $I_1 \subseteq I$, $I = \{ 1, \dots, m \}$, and $J=\{ 1, \dots, n \}$.
And $I_1$ and $J_1$ are the indices of the inequality constraints.
\end{definition}

\begin{remark}
We write $[A, b, c, I_1, J_1]$ to display the relevant information determining in a general linear programming problem \arefeq{LP_Po} defined in \refdef{general_LP}.

The canonical problem \arefeq{LP_P} and its dual \arefeq{LP_D} 
correspond to $J_1 = J, J_2 = \emptyset, I_1 = \emptyset, I_2 = I$.

The standard problem \arefeq{LP_Ps} and its dual \arefeq{LP_Ds} correspond to $J_1 = J, J_2 = \emptyset, I_1 = I, I_2 = \emptyset$.
\end{remark}

\begin{remark}
\labremark{d_o_general}
We could check that the duality and optimality theorem (\refthm{d_o}) applies to the general forms \arefeq{LP_Po} and \arefeq{LP_Do}.
\end{remark}

\begin{theorem}[Linear programming duality]
The dual of \arefeq{LP_Do} is \arefeq{LP_Po}.
\end{theorem}

\begin{proof}
The problem \arefeq{LP_Do} is equivalent to 
\begin{align}
\min \ & (-b)^T y & \\
\st    & \sum_{i=1}^{m} (-a_{ij}) y_i \ge c_j, & j \in J_1 \\
       & \sum_{i=1}^{m} (-a_{ij}) y_i = c_j, & j \in J_2 \\
       & y_i \ge 0 & i \in I_1
\end{align}
This is $[-A^T, -c, -b, J_1, I_1]$. 
So duality converts
\begin{align}
\underbrace{[A, b, c, I_1, J_1]}_{P^o} \to \underbrace{[-A^T, -c, -b, J_1, I_1]}_{D^o}.
\end{align}
Hence the dual of \arefeq{LP_Do} is
\begin{align}
[-(-A^T)^T, -(-b), -(-c), I_1, J_1] = [A, b, c, I_1, J_1].
\end{align}
\end{proof}

\begin{remark}
By adding new slack variables, we can in fact convert a general linear programming problem \arefeq{LP_Po} into the canonical form \arefeq{LP_P}.

As a consequence of these observations, when we study the theory of linear programming, it is enough to consider the canonical problem.
\end{remark}

\subsection{Equilibrium equations}

\begin{theorem}[Equilibrium equations]
Suppose $x$ is feasible for \arefeq{LP_P} and $y$ is feasible for \arefeq{LP_D}.
Then $x$ and $y$ are optimal iff they satisfy the equilibrium equations
\begin{align}
\sum_{i=1}^{m} y_i a_{ij} = c_j \text{ if } x_j > 0, j = 1,\dots,n. \tag{E} \labeq{LP_E}
\end{align}
And equivalently,
\begin{align}
\sum_{i=1}^{m} y_i a_{ij} < c_j \text{ if } x_j = 0, j = 1,\dots,n. \tag{E'}
\end{align}
\end{theorem}

\begin{proof}
As before, we compute $b^T y = (Ax)^T y = x^T A^T y \le x^T c$.
Our question is when do we have equality in the last inequality?
Note that 
\begin{align}
x^T (A^T y - c) = \sum_{j=1}^{n} x_j \left( (A^Ty)_j - c_j \right) \begin{cases}
= 0, \text{ if \arefeq{LP_E} holds} \\
< 0, \text{ if \arefeq{LP_E} fails}
\end{cases}
\end{align}
where $(A^T y)_j = \sum_{i=1}^{m} y_i a_{ij}$. 
Thus we have $b^T y = c^T x$ if \arefeq{LP_E} holds.
By \refthm{d_o}, \arefeq{LP_E} implies the optimality for both $x$ and $y$.
\end{proof}

\subsection{Basic solutions}

We introduce next the concept of basic solutions to linear programming problems.
These are solutions with the largest numbers of zero entries, which are consequently the easiest to study.

A $m \times n$ matrix $A$ can be written as $[a^1 a^2 \cdots a^n]$, where $a^j$ is the $j$-th column vector of $A$, and so $a^j \in \R^m$ for $j = 1, \dots, n$. 

\begin{lemma}
If $Ax = b$, then $b$ is a linear combination of the columns of $A$.
\end{lemma}

\begin{proof}
We can write $Ax=b$ as $\sum_{j=1}^{n} x_j a^j = b$, and this shows $b$ to be a linear combination of the column vectors of $A$.
\end{proof}

\begin{definition}
\labdef{LP_basic_solution}
\index{basic solutions}
We say that $x \in \R^n$ is a basic solution of $Ax = b$ if the columns $\{ a^j : x_j \neq 0, j=1,\dots,n \}$ are linearly independent in $\R^m$.
Also, we say $x = 0$ is basic.
\end{definition}



\begin{theorem}
\labthm{LP_finite_basic_solution}
For each $b \in \R^m$ the linear system of equations $Ax = b$ has at most finitely many basic solutions.
\end{theorem}

\begin{proof}
Look at columns $\{ a^1,\dots, a^n \}$ of $A$.
There are only finitely many subsets $\{ a^{j_1},\dots, a^{j_l} \} \subseteq \{ a^1,\dots, a^n \}$ which are independent in $\R^m$.
We claim there is at most one solution of $Ax=b$ having the form $x = [0 x_{j_1} \dots x_{j_l} 0]^T$.
To see this, let $\hat{x} = [0 \hat{x}_{j_1} \dots \hat{x}_{j_l} 0]^T$ also solve $A\hat{x} = b$.
Then $A(x-\hat{x}) = b -b = 0$ and therefore $\sum_{k=1}^{l} (x_{j_k} - \hat{x}_{j_k})a^{j_k} = 0$.
Since the columns $\{ a^{j_1},\dots, a^{j_l} \}$ are linearly independent, it follows that $x_{j_k} = \hat{x}_{j_k}$.
Hence there is at most one basic solution of $Ax = b$ corresponding to each
independent collection of columns.
\end{proof}

\begin{theorem}[Basic solutions]
\labthm{LP_basic_solution}
If there exists a feasible solution of \arefeq{LP_P}, then there exists a basic feasible solution.

If there exists an optimal solution of \arefeq{LP_P}, then there exists a basic optimal solution. 
\end{theorem}

\begin{proof}
Step 1:
Select a feasible solution $x$ with fewest number of nonzero components.
And then let us show it is a basic feasible solution.

If $x=0$, we are done.
If not, then suppose that $x$ has nonzero elements $x_{j_1}, x_{j_2}, \dots, x_{j_l} > 0$ and 
\begin{align}
Ax = \sum _{k=1}^{l} x_{j_k}a^{j_k} = b. \labeq{proof_basic_sol_1}
\end{align}
Then suppose that $x$ is not basic. 
By \refdef{LP_basic_solution}, $\{ a^{j_1},\dots, a^{j_l} \}$ are not linearly independent.
There exists $\theta_{j_1}, \theta_{j_2}, \dots, \theta_{j_l}$ not all equal to zero, such that
\begin{align}
\sum_{k=1}^{l} \theta_{j_k} a^{j_k} = 0. \labeq{proof_basic_sol_2}
\end{align}
This means that $A\theta = 0$.
Then \refeq{proof_basic_sol_1} and \refeq{proof_basic_sol_2} simply imply for any $\lambda$ that
\begin{align}
\sum _{k=1}^{l} (x_{j_k} - \lambda \theta_{j_k}) a^{j_k} = b.
\end{align}
We may assume $\theta_{j_p} > 0$ for some index $j_p$ (if not, multiply $\theta$ by $-1$).
Increase $\lambda$ from $0$ to the first $\lambda^*>0$ for which at least one of the values $x_{j_1} - \lambda^* \theta_{j_1}, \dots, x_{j_l} - \lambda^* \theta_{j_l}$ equals zero while others remain to be greater than zero.
Since $\theta_{j_p}>0$, this must happen at finite value of $\lambda^*$.
Then let $x^* = x - \lambda^* \theta \ge 0$ and $A x^* = 0$.
But $x^*$ then has at least one fewer nonzero entry than $x$.
This is a contradiction, and therefore $x$ is indeed a basic feasible solution.

Step 2:
Now let $x_0$ be an optimal solution with the fewest number of nonzero components. 
We will show that $x_0$ is a basic optimal solution.

Suppose $x_0$ is not.
Then as above, suppose $x_0$ has nonzero elements $x_{j_1}, x_{j_2}, \dots, x_{j_l} > 0$ and $\sum _{k=1}^{l} x_{j_k}a^{j_k} = b$ and $\sum_{k=1}^{l} \theta_{j_k} a^{j_k} = 0$ for appropriate $\theta_{j_1}, \dots, \theta_{j_l}$ not all zero.
Select $\lambda^*$ as before and write $x_0^* = x_0 - \lambda^* \theta$.
Then $A x_0^* = b$, $x_0^* \ge 0$, and $x_0^*$ has fewer nonzero components than $x_0$ which leads to contradiction of that $x_0$ has the fewest number of nonzero components.

Step 3:
We now claim
\begin{align}
c^T x_0^* = c^T x_0 = \min \{ c^T x : Ax=b, x\ge 0 \}. \labeq{proof_basic_sol_3}
\end{align}
To prove this, observe first that 
\begin{align}
c^T \theta = 0; \labeq{proof_basic_sol_4}
\end{align}
for otherwise, we could select a small value of $\lambda$ so that $c^T (x_0 - \lambda \theta) < c^T x_0$ and $x_0$ is not optimal consequently.
Thus \refeq{proof_basic_sol_4} must hold and therefore \refeq{proof_basic_sol_3} holds. 
Thus $x_0^*$ is optimal for \arefeq{LP_P}, but has fewer nonzero components than $x_0$.
And this is a contradiction: $x_0$ is a basic optimal solution.
\end{proof}

\begin{remark}
Our discussion of basic solutions leads to the very interesting realization that although linear programming problems are finite dimensional, with infinitely many feasible solutions, they are in effect finite optimization problems, with only finitely many basic solutions to consider and only finitely many optimal basic solutions.

The simplex algorithm, discussed next, builds upon this observation.
\end{remark}

\section{Simplex algorithm}
\index{simplex algorithm}
The simplex algorithm comprises two procedures:

Phase I: Find a basic feasible solution of $Ax = b$, $x \ge 0$ (or show that none exists).

Phase II: Given a basic feasible solution, find a basic optimal solution (or show that none exists).

\subsection{Nondegeneracy}
\index{nondegeneracy}
\begin{assumption}[Nondegeneracy]
\labassum{nondegeneracy}
The nondegeneracy assumptions are that
\begin{enumerate}
    \item $n > m$;
    \item the rows of $A$ are linearly independent (and thus $A$ has $m$ columns which are independent);
    \item $b$ cannot be written as a linear combination of fewer than $m$ columns of $A$.
\end{enumerate}
\end{assumption}

Assumption 1 implies that there are more unknowns ($x_1, \dots , x_n$) than the $m$ linear equality constraints in the linear system $Ax = b$.

Assumption 2 means that $\mathrm{rank}(A) = \dim (\text{column space}) = \dim (\text{row space}) = m$.

Assumption 3 says that if $Ax = b$, then $x$ has at least $m$ nonzero entries.

\begin{remark}
Under the nondegeneracy assumptions, any basic, feasible solution of $Ax = b, x \ge 0$ has precisely $m$ nonzero entries.
The next assertion shows that the converse is true as well
\end{remark}

\begin{lemma}[On nondegeneracy]
\lablemma{on_nondegeneracy}
Assume the nondegeneracy assumptions 1-3 hold.
If $Ax = b, x \ge 0$ and $x$ has precisely $m$ non-zero entries, then $x$ is a basic feasible solution.
\end{lemma}

\begin{proof}
Let $x = [0 x_{j_1} 0 \dots x_{j_m} \dots 0]^T$, where $x_{j_1}, \dots x_{j_m} > 0$, and write $B = \{ j_1, \dots, j_m \}$.
To show that $x$ is a basic feasible solution, We must prove that the columns $\{ a^{j_1}, \dots, a^{j_m} \}$ are independent. 

We know that $\sum _{j \in B} x_j a^j = b$.
If the columns $\{ a^{j_1}, \dots, a^{j_m} \}$ are dependent, we could write some column as a linear combination of the others.
This is, for some index $j_k$ we have $a^{j_k} = \sum _ {j \in B, j \neq j_k} y_j a^j$.
Then 
\begin{align}
b = x_{j_k} a^{j_k} + \sum _ {j \in B, j \neq j_k} x_j a^j 
= x_{j_k} \sum _ {j \in B, j \neq j_k} y_j a^j + \sum _ {j \in B, j \neq j_k} x_j a^j 
= \sum _ {j \in B, j \neq j_k} (x_{j_k} y_{j} + x_j) a^j.
\end{align}
Thus $b$ is a linear combination of few than $m$ columns of $A$, a contradiction to the nondegeneracy requirement 3.
\end{proof}

\subsection{Phase II}

We discuss Phase II before Phase I.
The goal of Phase II is, given a basic feasible solution $x$, to find a basic optimal solution $x_0$, or show none exists.
For this, we assume the nondegeneracy condition 1, 2, 3.

So we are given $x = [0 \ x_{j_1} \ 0 \dots x_{j_m} \dots 0]^T$ where $x_{j_1}, \dots x_{j_m} > 0$ are the $m$ nonzero entries of $x$.
We also have $Ax = b$.

\subsubsection{Step 1: Use the dual problem to check for optimality}

We have a basic feasible solution $x$, and need to check if it is optimal or not.

\begin{definition}
Suppose we have a basic feasible solution $x$.
Let $B = \{ j : x_j > 0 \} = \{ j_1, \dots, j_m \}$. 
We call $\{ a^j: j \in B \}$ the basis corresponding to $x$.

The $m \times m$ matrix $M = [a^{j_1} \ a^{j_2} \cdots a^{j_m}]_{m \times m}$, is called the corresponding basis matrix.

If $c = [c_1 \dots c_n]^T$, define $\hat{c} = [c_{j_1} \dots c_{j_m}]^T \in \R^m$.
\end{definition}

\begin{remark}
The matrix $M$ is invertible since its columns are independent.
Thus there exist a unique $y \in \R^m$ solving $M^T y = \hat{c}$ and $y = (M^{-1})^T \hat{c}$.

Recall next that $y$ is feasible for \arefeq{LP_D} if $A^T y \le c$.
This may or may not hold.
But if it holds, we are done.
\end{remark}

\begin{lemma}
If $A^T y \le c$ holds, then $x$ is optimal for \arefeq{LP_P}.
\end{lemma}

\begin{proof}
The equilibrium equations \arefeq{LP_E} say 
\begin{align}
(a^j)^T y = \sum _{i=1}^{m} y_i a_{ij} = c_j \text{ if } x_j > 0 (\text{or} j \in B).
\end{align}
Now $M = [a^{j_1} \dots a^{j_m}]$, and so
\begin{align}
M^T = \begin{bmatrix}
(a^{j_1})^T \\
\vdots \\
(a^{j_m})^T \\
\end{bmatrix}
\end{align}
But since there exists a $y$ such that $M^T y = \hat{c}$, which means that $(a^j)^T y = c_j$ for $j \in B$.
These are precisely the equilibrium question \arefeq{LP_E}. 
So if $y$ is feasible for \arefeq{LP_D}, it follows that $x$ is optimal for \arefeq{LP_P} and $y$ is optimal for \arefeq{LP_D}.
\end{proof}

Therefore we have two possibilities:
\begin{enumerate}
    \item[A1:] $y = (M^{-1})^T \hat{c}$ satisfies $A^T y \le c$.
    Then stop: $x$ is optimal for \refeq{LP_P} and $y$ is optimal for \arefeq{LP_D}.
    \item[A2:] $y = (M^{-1})^T \hat{c}$ does not satisfy $A^T y \le c$.
    Go to step 2.
\end{enumerate}

\subsubsection{Step 2: Use a wrong way inequality to improve $x$}

When $A^T y \le c$ fails, there exists some index $s \in (\{ 1, \dots, n \} - B)$ such that 
\begin{align}
\underbrace{(a^s)^T y > c_s.}_{\text{wrong way inequality}} \labeq{LP_wrong_way}
\end{align}
The key idea of the simplex algorithm is use this fact to change the basis $\{a^j : j \in B \}$, thereby constructing a new basic feasible solution $x^*$ with a lower cost $c^T x^*$.
To do this, we first find $t = [t_{j_1} \dots t_{j_m}] \in \R^m$ so that $Mt = a^s$.
Since $M$ is invertible, we have a unique solution $t$.
Then $a^s = \sum _{j \in B} t_j a^j$, and consequently for all $\lambda$ we have $\lambda a^s + \sum _{j \in B} (x_j - \lambda t_j) a^j = b$.
We then define
\begin{align}
\hat{x} = \begin{bmatrix}
0 \\
x_{j_1} - \lambda t_{j_1} \\
\vdots \\
x_{j_2} - \lambda t_{j_2} \\
\vdots \\
\lambda \\
\vdots \\
x_{j_m} - \lambda t_{j_m} \\
0 \\
\end{bmatrix}
\in \R^n.
\end{align}

Here $\lambda$ is in the $s$-th slot.
And for a small $\lambda > 0$, we have $A\hat{x} = b$.
Thus $\hat{x}$ is feasible for \arefeq{LP_P} small $\lambda > 0$.

However, we must note that $\hat{x}$ has $m+1$ nonzero entries, and consequently is not basic.

How does replacing $x$ by $\hat{x}$ affect the cost?
We would be pleased if $c^T \hat{x} < c^T x$ after replacing $x$ by $\hat{x}$.
\begin{align}
c^T \hat{x} - c^T x 
&= \lambda c_s + \sum _{j \in B} (x_j - \lambda t_j) c_j - \sum _{j \in B} x_j c_j \\
&= \lambda c_s - \lambda \underbrace{\sum _{j \in B} t_j c_j}_{z_s} \\
&= \lambda (c_s - z_s).
\end{align}
And $z_s = \sum _{j \in B} t_j c_j = \hat{c}^T t = \hat{c}^T (M^-1 a^s) = y^T a^s$.
Therefore \arefeq{LP_wrong_way}  is equivalent to $z_s > c_s$.
It follows that $c^T \hat{x} < c^T x$.
Consequently, we lower the cost by shifting to $\hat{x}$ from $x$.

\subsubsection{Step 3: Change the basis, lower the cost}

There are two possibilities as to how much we can lower the cost by increasing $\lambda$:

\begin{enumerate}
    \item[B1:] $t_j \le 0$ for all $j \in B$.
    Then $\hat{x}_j = x_j - \lambda t_j \ge x_j > 0$ for $j \in B$ and so $\hat{x}$ is feasible for all $\lambda > 0$.
    Thus  says
    \begin{align}
        c^T \hat{x} = c^T x + \lambda \underbrace{(c_s - z_s)}_{< 0} \to - \infty
    \end{align} as $\lambda \to \infty$.
    So we have learned that $\inf \{c^T x : Ax = b, x \ge 0 \} = -\infty$ and therefore stop and concluded that \arefeq{LP_P} has no solution.
    \item[B2:] $t_j > 0$ for at least one index $j \in B$.
    We increase $\lambda$, starting at $0$ and stopping when $\lambda = \lambda^* > 0$ and $\hat{x}_{j_k} = x_{j_k} - \lambda^* t_{j_k} = 0$ for some index $j_k \in B$. 
    Define that 
    \begin{align}
        x^* = \begin{bmatrix}
            0 \\
            x_{j_1} - \lambda^* t_{j_1} \\
            \vdots \\
            x_{j_k} - \lambda^* t_{j_k} (= 0) \\
            \vdots \\
            \lambda^* \\
            \vdots \\
            x_{j_m} - \lambda^* t_{j_m} \\
            0 \\
        \end{bmatrix}.
    \end{align}
    Then $x^*$ has no more than $m$ nonzero entries; and, since $Ax^*=b$, the nondegeneracy condition say that $x^*$ has precisely $m$ nonzero entries.
    According to \reflemma{on_nondegeneracy}, $x^*$ is therefore a basic feasible solution.
    Furthermore, $c^T x^* < c^T x$.
    
    Now define the new basis 
    \begin{align}
    B^* = (\underbrace{\{ j_1, \dots, j_m \}}_{B} - \{ j_k \}) \cup \{ s \},
    \end{align}
    by removing the index $j_k$ and adding the index $s$. 
    Then go to step 1 again with $x^*$ replacing $x$ and $B^*$ replacing $B$.
\end{enumerate}

Each time we cycle through step 1 to step 3, the cost strictly decreases.
Thus the same collection of basis vectors $B$ will never repeat.
Hence the simplex algorithm terminates in a finite number of step.
This can only occur when A1 happens which suggests we have reached an optimal solution or B1 happens which means that none exists.

\begin{theorem}[Simplex algorithm finds optimal solutions]
Assume the nondegeneracy conditions 1, 2, 3 hold, and that there exist feasible $x$ for \arefeq{LP_P}, feasible $y$ for \arefeq{LP_D}.
Then the simplex algorithm terminates in finitely many steps, and produces a basic optimal $x_0$ for \arefeq{LP_P} and an basic optimal $y_0$ for \arefeq{LP_D}.
\end{theorem}

\begin{proof}
If $y$ is feasible for \arefeq{LP_D}, $\inf \ \{ c^T x : Ax = b, x \ge 0 \} \ge b^T y > -\infty$,
and so B1 cannot occur.
Consequently, the simplex algorithm terminates at an optimal $x_0$ for \arefeq{LP_P} and $y_0$ for \arefeq{LP_D}.
\end{proof}

\subsection{Phase I}

We now come back and explain how to carry Phase I of the simplex algorithm.
This major focus of Phase I is to find a feasible solution of $Ax = b, x \ge 0$.
The following discussions are trying to convert Phase II to Phase I.
And in this way, we will not include other unproved definitions or theorems.
However, alternatively we can do Gauss elimination.

\begin{remark}
To make a connection between Phase I and II, we need to modify the third nondegeneracy
condition, to become 
\begin{enumerate}
    \item[3'.] $b$ cannot be written as a linear combination of fewer than $m$ columns of $\tilde{A} = [A \ I_m]$, where $I_m$ is $m \times m$ identity matrix.
\end{enumerate}
Note that 3' indicates $b_i \neq 0$ for all $i = 1, \dots, m$.
\end{remark}

We assume for this section that the nondegeneracy conditions 1, 2, and 3' hold.

The goal of Phase I is to find $x \ge 0$ solving $Ax = b$, that is,
\begin{align}
\sum_{j = 1}^{n} a_{ij} x_j = b_i, (i = 1, \dots, m).
\end{align}
We may assume that $b_i > 0$ for otherwise we can multiply $i$-th constraint by $-1$.

Now let $z$ be a vector of $m$ elements and consider the modified problem:
\begin{align}
\min \ & z \tag{$\tilde{P}$} \labeq{LP_tildeP} \\
\st    & Ax + I_m z = b \nonumber \\
       & x \ge 0 \nonumber \\
       & z \ge 0. \nonumber
\end{align}
This has the form
\begin{align*}
\min \ & \tilde{c}^T \tilde{x} \\
\st    & \tilde{A} \tilde{x} = \tilde{b} \\
       & \tilde{x} \ge 0
\end{align*}
where $\tilde{x} = [x \ z]^T$, $\tilde{A} = [A \ I_m]$, $\tilde{b} = b$, and $\tilde{c} = [\underbrace{0 \dots 0}_{n} \ \underbrace{1 \dots 1}_{m}]^T$.

Since each $b_i > 0$, a basic feasible solution of \arefeq{LP_tildeP} is $\tilde{x} = [0 \dots 0 \ b_1 \dots b_m]^T$.
Now apply Phase II to \arefeq{LP_tildeP}: we either produce a basic optimal solution $\tilde{x}_0$ of \arefeq{LP_tildeP} or learn that none exists.
Since $\tilde{c}^T \tilde{x} \geq 0$ for all feasible $\tilde{x}$, the latter cannot occur, as we will later see from .
Hence Phase II provides use with a basic optimal $\tilde{x}_0 = [x \ z]^T$ for \arefeq{LP_tildeP}.

There are now two possibilities to consider:
\begin{enumerate}
    \item[C1:] $\sum_{i=1}^{m} z_i = 0$.
    Then $z = 0$, and therefore $\tilde{A} \tilde{x} = \tilde{b}$ implies $Ax = b, x \ge 0$.
    So we have a basic feasible solution $x$ for \arefeq{LP_P}.
    \item[C2:] $\sum_{i=1}^{m} z_i > 0$.
    In this situation, \arefeq{LP_P} does not have any feasible solution $x$.
    This is so, since if $Ax = b, x \ge 0$, then $\tilde{x}_0 = [x \ 0]^T$ would be optimal for \arefeq{LP_tildeP}, giving the cost $\tilde{c}^T \tilde{x} = \sum_{i=1}^{m} z_i = 0$.
\end{enumerate}

\section{Duality Theorem}

Next we return to theory and provide an analysis of the solvability of linear programming problems in standard form \arefeq{LP_Ps} and \arefeq{LP_Ds}.

We no longer need the nondegeneracy conditions from the previous section, but we do require this important assertion:

\begin{theorem}[Variant of Farkas alternative]
% \index{Variant of Farkas alternative}
\labthm{Farkas_alt_0}
Either 
\begin{enumerate}
    \item $Ax \le b, x \ge 0$ has a solution; or
    \item $A^T y \ge 0, b^T y < 0, y \ge 0$ has a solution,
\end{enumerate}
but not both.
\end{theorem}

The formal proof of this theorem is postponed to \vrefthm{farkas_alternative}.

\begin{theorem}[Duality Theorem for standard form problems]
\index{Duality Theorem for standard form problems}
Precisely one of the following occurs:
\begin{enumerate}
    \item Both \arefeq{LP_Ps} and \arefeq{LP_Ds} have feasible solutions.
    In this case, both \arefeq{LP_Ps} and \arefeq{LP_Ds} have optimal solutions and
    \begin{align}
    \min \ \{ c^T x : Ax \ge b, x \ge 0 \} = \max \ \{ b^T y : A^T y \le c, y \ge 0 \}.
    \end{align}
    \item There are feasible solutions for \arefeq{LP_Ds}, but not for \arefeq{LP_Ps}. 
    Then
    \begin{align}
    \sup \ \{ b^T y : A^T y \le c, y \ge 0 \} = \infty.    
    \end{align}
    \item There are feasible solutions for \arefeq{LP_Ps}, but not for \arefeq{LP_Ds}. 
    Then 
    \begin{align}
    \inf \ \{ c^T x : Ax \ge b, x \ge 0 \} = -\infty.
    \end{align}
    \item Neither \arefeq{LP_Ps} nor \arefeq{LP_Ds} has feasible solutions.
    \end{enumerate}
\end{theorem}

\begin{proof}
Step 1:
We introduce the compound matrix
\begin{align}
\hat{A} &= \begin{bmatrix}
-A & 0 \\
0  & A^T \\
c^T & -b^T \\
\end{bmatrix}_{(m+n+1) \times (n+m)} \\
\hat{x} &= \begin{bmatrix}
x \\
y \\
\end{bmatrix} \\
\hat{b} &= \begin{bmatrix}
-b \\
c  \\
0 \\
\end{bmatrix} \\
\hat{y} &= \begin{bmatrix}
v \\
u \\
\lambda \\
\end{bmatrix}.
\end{align}

Then \refthm{Farkas_alt_0} states that either
\begin{enumerate}
    \item $\hat{A} \hat{x} \le \hat{b}, \hat{x} \ge 0$ has a solution; or 
    \item $\hat{A}^T \hat{y} \ge 0, \hat{b}^T \hat{y} < 0, \hat{y} \ge 0$ has a solution
\end{enumerate}
but not both.

Step 2:
If 1 holds, which means $\hat{A} \hat{x} \le \hat{b}, \hat{x} \ge 0$ has a solution, then there exist $x \in \R^n$ and $y \in \R^m$ so that $x \ge 0, y \ge 0$ and 
\begin{align}
\begin{bmatrix}
-A & 0 \\
0  & A^T \\
c^T & -b^T \\
\end{bmatrix}
\begin{bmatrix}
x \\
y \\
\end{bmatrix}
\le 
\begin{bmatrix}
-b \\
c  \\
0 \\
\end{bmatrix}.
\end{align}
Thus 
\begin{align}
-A x \le -b \\
A^T y \le c \\
c^T x - b^T y \le 0 \\
x \ge 0 \\
y \ge 0
\end{align}
Consequently, we have $c^T x \le b^T y \le (Ax)^T y = x^T A^T y \le x^T c$ which implies $c^T x = b^T y$.
Hence
\begin{align}
A x \ge b \\
A^T y \le c \\
c^T x = b^T y  \\
x \ge 0 \\
y \ge 0
\end{align}
Therefore, if $x$ is feasible for \arefeq{LP_Ps} and $y$ is feasible for \arefeq{LP_Ds}, then according to \refthm{d_o} and \refremark{d_o_standard}, $c^T x = b^T y$ implies that $x$ is optimal for \arefeq{LP_Ps} and $y$ is optimal for \arefeq{LP_Ds}.
This is the statement 1 of Duality Theorem.

Step 3:
If 2 holds, which means $\hat{A}^T \hat{y} \ge 0, \hat{b}^T \hat{y} < 0, \hat{y} \ge 0$ has a solution, then there exist $v \in \R^m, u \in \R^n, \lambda \in \R$ with $v, u, \lambda \ge 0$,
\begin{align}
\begin{bmatrix}
-A^T & 0 & c \\
0 & A & -b \\
\end{bmatrix}
\begin{bmatrix}
v \\
u \\
\lambda \\
\end{bmatrix}
\ge 
\begin{bmatrix}
0 \\
0 \\
0 \\
\end{bmatrix},
\end{align}
and 
\begin{align}
\hat{b}^T \hat{y} = [-b^T c^T 0] 
\begin{bmatrix}
v \\
u \\
\lambda \\
\end{bmatrix}
= -b^T v + c^T u \le 0.
\end{align}
That is
\begin{align}
-A^T v + c \lambda \ge 0 \\
A u - b \lambda \ge 0 \\
-b^T v + c^T u < 0\\
v \ge 0 \\
u \ge 0 \\ 
\lambda \ge 0 
\end{align}
Next, we assert that $\lambda = 0$.
Observe that $\lambda (b^T v) \le u^T A^T v \le u^T (c \lambda) = \lambda (c^T u)$, so $\lambda (-b^T v + c^T u) \ge 0$.
But as $\lambda \ge 0$, it contradicts $-b^T v + c^T u < 0$ if $\lambda > 0$.
So we have $\lambda = 0$ and 
\begin{align}
A^T v \le 0 \\
A u \ge 0 \\
c^T u \le b^T v \\
v \ge 0 \\
u \ge 0.
\end{align}
The existence of $u, v$ satisfying 2 leads us to statements 2-4 of Duality Theorem.
The next step is to discuss the positiveness of $c^T u$.

Step 4:
If $c^T u < 0$, \arefeq{LP_Ds} has no feasible solution.
For $A^T y \le c, y \ge 0$, 2 implies $0 \le y^T A u = (A^T y)^T u \le c^T u < 0$, which is a contradiction.

If in addition \arefeq{LP_Ps} has no feasible solution, we have statement 4 of the Duality Theorem.
If, on the other hand, \arefeq{LP_Ps} has a feasible solution $x$ solving $A x \ge b, x \ge 0$ then for all $\mu \ge 0$, we have
\begin{align}
A (x + \mu u) = A x + \mu A u \ge b \\
x + \mu u \ge 0
\end{align}
So $x + \mu u$ is also feasible for \arefeq{LP_Ps}. 
But then since $c^T u < 0$, $c^T (x + \mu u) = c^T x + \mu c^T u \to -\infty$ as $\mu \to \infty$.
This is statement 3 of the Duality Theorem.

Step 5:
If $c^T u \ge 0$, \arefeq{LP_Ps} has not feasible solution.
Assume $A x \ge b, x \ge 0$.
Following 2, then $0 \ge x^T A^T v = (Ax)^T v \ge b^T v > c^T u \ge 0$, which is a contradiction.

Also if \arefeq{LP_Ds} has no feasible solution, we have statement 4 of the Duality Theorem.
If \arefeq{LP_Ds} has a feasible solution $y$ solving $A^T y \le c, y \ge 0$, then for all $\mu \ge 0$, we have
\begin{align}
A^T (y + \mu v) = A^T y + \mu A^T v \le c \\
y + \mu v \ge 0
\end{align}
So $y + \mu v$ is also feasible for \arefeq{LP_Ds}.
But then since $b^T v > c^T u \ge 0$, $b^T (y + \mu v) = b^T y + \mu b^T v \to \infty$ as $\mu \to \infty$.
This is the statement 2 of the Duality Theorem.
\end{proof}

Let us now return to the canonical forms of our primal and dual problems: \arefeq{LP_P} and \arefeq{LP_D}.
Since we can use slack and surplus variables to convert between the canonical and the standard form problems \arefeq{LP_Ps} and \arefeq{LP_Ds}, we like wise have a duality assertion for the canonical problems.

\begin{theorem}[Duality Theorem for canonical form problems]
\index{Duality Theorem for canonical form problems}
Precisely one of the following occurs:
\begin{enumerate}
    \item Both \arefeq{LP_P} and \arefeq{LP_D} have feasible solutions.
    In this case, both \arefeq{LP_P} and \arefeq{LP_D} have optimal solutions and
    \begin{align}
    \min \ \{ c^T x : Ax = b, x \ge 0 \} = \max \ \{ b^T y : A^T y \le c \}.
    \end{align}
    \item There are feasible solutions for \arefeq{LP_D}, but not for \arefeq{LP_P}. 
    Then
    \begin{align}
    \sup \ \{ b^T y : A^T y \le c \} = \infty.    
    \end{align}
    \item There are feasible solutions for \arefeq{LP_P}, but not for \arefeq{LP_D}. 
    Then 
    \begin{align}
    \inf \ \{ c^T x : Ax = b, x \ge 0 \} = -\infty.
    \end{align}
    \item Neither \arefeq{LP_P} nor \arefeq{LP_D} has feasible solutions.
\end{enumerate}
\end{theorem}

\marginnote{
\begin{remark}
What about general forms?
\end{remark}
}

\section{Applications}

We discuss in the subsequent sections several interesting applications and extensions of linear programming.

\subsection{Two-person, zero-sum matrix games}

In a two-person, zero-sum matrix game, we have two participants: player I (who wants to maximize some payoff) and player II (who wants to minimize this payoff). 
Each player selects his/her strategy without knowing what the
other will do.

For a matrix game, the payoff is determined by a given $m \times n$ payoff matrix:

\begin{align}
    \begin{bmatrix}
        a_{11} & \cdots &        &        & a_{1n} \\
        \vdots & \ddots &        &        & \\
               &        & a_{ij} &        & \\
               &        &        & \ddots & \vdots \\
        a_{m1} &        &        & \cdots & a_{mn} \\
    \end{bmatrix}
\end{align}

Player I selects a row index $i \in \{ 1, \dots, m \}$, and player II selects a column index $j \in \{ 1, \dots, n \}$.
The payoff to player I is $a_{ij}$ and the loss to player II is $a_{ij}$.

What are optimal strategies for the players?

\begin{definition}[Saddle point]
\labdef{saddle_point}
\index{Saddle point}
The $(k,l)$-th entry $a_{kl}$ of the matrix $A$ is a saddle point if 
\begin{align}
    \max _{1 \le i \le m} a_{il} = a_{kl} = \min _{1 \le j \le n} a_{kj}.
\end{align}
Equivalently, $a_{kl}$ is a saddle point if 
\begin{align}
    a_{il} \le a_{kl} \le a_{kj}
\end{align}
for all $1 \le i \le m$ and $1 \le j \le n$.
\end{definition}

\begin{theorem}[Minmax and saddle points]
    \labthm{minmax_saddle_points}
    The matrix A has a saddle point iff the minimax condition
    \begin{align}
        \min _{1 \le j \le n} \max _{1 \le i \le m} a_{ij} = \max _{1 \le i \le m} \min _{1 \le j \le n} a_{ij}
    \end{align}
    holds.
\end{theorem}

\begin{example}
    For the matrix $A = \begin{bmatrix} 1 & -1 \\ -1 & 1 \\ \end{bmatrix}$, we have $\min _{j} \max _{i} a_{ij} = 1$ and $\max _{i} \min _{j} a_{ij} = -1$.
    Therefore the minimax condition (\refthm{minmax_saddle_points}) fails.
    So $A$ does not have a saddle points.
\end{example}

\subsection{Network flows}

TODO

\subsection{Transportation problem}

TODO
\setchapterstyle{kao}
\setchapterpreamble[u]{\margintoc}
\chapter{Convexity}
\labch{convexity}

\section{Convex geometry}

\subsection{Convex sets}

\begin{definition}
A set $C \subseteq \R^n$ is convex if for all $a, b \in C$ and $0 \le \theta \le 1$, we have $\theta a + (1-\theta) b \in C$.
\end{definition}

If two points belong to a convex set, then all points that are on the line between these two points belong to this convex set.

\begin{definition}[Convex combination]
Let $\{ a^1, \dots, a^p \} \subset \R^n$.
A convex combination of $a^1, \dots a^p$ is $\sum _{k=1}^{p} \theta_{k} a^k$ for some $\theta_k \ge 0$ and $\sum _{k=1}^{p} \theta_k = 1$.
\end{definition}

The convex combination can be regarded as a linear combination with restraints on coefficients $\theta_k$.

\begin{definition}[Convex polytope]
The convex polytope generated by $a^1, \dots, a^p$ is $\langle a^1, \dots, a^p \rangle = \left\{ \sum _{k=1}^{p} \theta_{k} a^k : \theta_k \ge 0, \sum _{k=1}^{p} \theta_k = 1 \right\}$
\end{definition}

A convex polytope is a set of all possible convex combinations.

\begin{theorem}[Caratheodory's Theorem]
Let $b$ belong to the convex polytope $\langle a^1, \dots, a^p \rangle \subset \R^n$.
Then we can write $b = \sum _{k=1}^{n+1} \theta_k a^{j_k}$ where $1 \le j_1 < \cdots < j_{n+1} \le p, \theta_k \ge 0, \sum _{k=1}^{n+1} \theta_k = 1$.
\end{theorem}

\begin{proof}
Since $b \in \langle a^1, \dots, a^p \rangle$, there exists a solution $x \in \R^p$ of 
\begin{align}
A x = \begin{bmatrix}
b \\
1 \\
\end{bmatrix},
x \ge 0
\end{align}
for the $(n+1) \times p$ matrix 
\begin{align}
A = \begin{bmatrix}
a^1 & a^2 & \cdots & a^p \\
1   & 1   & \cdots & 1   \\
\end{bmatrix}.
\end{align}
And according to \refthm{LP_basic_solution}, there exists a basic solution $x^* \in \R^p$ of 
\begin{align}
A x^* = \begin{bmatrix}
b \\
1 \\
\end{bmatrix},
x^* \ge 0.
\end{align}
This means that $x^*$ has at most $n+1$ nonzero entries $\{ x_{j_1}, \dots, x_{j_{n+1}} \}$ corresponding to the independent rows of $A$.
Then relabel $\theta_k = x_{j_k}$ for $k = 1, \dots, n+1$.
\end{proof}

\subsection{Separating hyperplanes}

We discuss now the geometry of convex sets and of hyperplanes. 

\begin{lemma}
\lablemma{separating_hyperplane}
Let $C$ be a nonempty, closed, convex, subset of $\R^n$ and suppose $0 \notin C$. 
Then there exists a unique point $x_0 \in C$ such that $|x_0| = \min \ \left\{ |x| : x \in C \right\}$.
Furthermore, $0 \le x_0 ^T (x-x_0)$ for all $x \in C$.
\end{lemma}

\begin{proof}
Step 1:
Let $\delta = \inf \ \{ |x| : x \in C \} \ge 0$.
Select $\{ x^k \}_{k=1}^{\infty} \subset C$ with $\delta = \lim _{k \to \infty} |x^k|$.
According to Bolzano-Weierstrass Theorem, there exists a convergent subsequence $\lim _{j \to \infty} x^{k_j} = x_0$.
Consequently, $|x_0| = \lim _{j \to \infty} x^{k_j} = \delta$.
Since $C$ is closed, $x_0 \in C$.
And since $0 \notin C$, $\delta > 0$.

Step 2:
We claim that $x_0$ is a unique point in $C$ with $|x_0| = \delta$.
Too see this, suppose $x_1 \in C$  also satisfies $|x_1| = \delta$. Then $\frac{x_0 + x_1}{2} \in C$ and $\left|\frac{x_0 + x_1}{2}\right| \ge \delta$.
However,
\begin{align}
\underbrace{|x_0 - x_1|^2}_{\ge 0} + \underbrace{|x_0 + x_1|^2}_{\ge 4 \delta^2} = \underbrace{2 (|x_0|^2 + |x_1|^2)}_{=4 \delta^2}.
\end{align}
Thus we must have $x_0 = x_1$.

Step 3:
Let $x$ be any point in $C$.
Then for $0 < \theta < 1$, $(1-\theta) x_0 + \theta x \in C$.
Therefore,
\begin{align}
|x_0|^2 & \le |(1-\theta) x_0 + \theta x|^2 \\
        & = |x_0 + \theta (x-x_0)|^2 \\
        & = |x_0|^2 + 2 \theta x_0^T (x-x_0) + \theta ^2 |x-x_0|^2 \\
0       & \le 2 x_0^T (x-x_0) + \theta |x-x_0|^2
\end{align}
Sending $\theta \to 0$, we have $0 \le x_0 ^T (x-x_0)$.
\end{proof}

\begin{definition}[Hyperplane]
Let $a \in \R^n, b \in \R$.
An expression of the form $a^T x + b = 0$ determines a hyperplane in $\R^n$.
The hyperplane is $\{ x : \in \R^n : a^T x + b = 0 \}$.
\end{definition}

It is an $(n-1)$-dimensional affine subspace and passes through the origin if and only if $b = 0$.

\begin{definition}
Let $S_1, S_2$ be two subsets of $\R^n$.
The hyperplane $a^T x + b = 0$ separates $S_1$ and $S_2$ if $a^T x + b \ge 0$ for all $x \in S_1$ and $a^T x + b \le 0$ for all $x \in S_2$.
The hyperplane $a^T x + b = 0$ strictly separates $S_1$ and $S_2$ if $a^T x + b > 0$ for all $x \in S_1$ and $a^T x + b < 0$ for all $x \in S_2$.
\end{definition}

\begin{theorem}[Separating Hyperplane Theorem]
\labthm{separating_hyperplane}
Let $C$ be convex, closed and non-empty, and suppose $e \notin C$.
Then there exists a hyperplane $a^T x + b$ that strictly separates $C$ and $\{ e \}$.
\end{theorem}

\begin{remark}
It is important for subsequent applications that in Separating Hyperplane Theorem, we do not require that $C$ be bounded.
\end{remark}

\begin{proof}
Step 1:
Upon shifting the coordinates if necessary, we may assume $e = 0$.
According to \reflemma{separating_hyperplane}, there exists $x_0 \in C$ such that $0 < \delta = |x_0| = \min \ \{ |x| : x \in C \}$.
We construct the separating hyperplane by finding a perpendicular bisector of the segment $\overline{e x_0}$.
Let $m = x_0 / 2$ and $a = x_0 / \delta$, then $a^T (x - m) = 0$ is the separating hyperplane.
Or equivalently, $a^T x + b = 0$ where $b = -a^T m$.

Step 2:
For $e=0$, $a^T (0 - m) = -\frac{|x_0|^2}{2 \delta} < 0$.

For $C$, taking $x \in C$, we have
\begin{align}
a^T (x-m) 
& = \frac{1}{\delta} x_0^T (x- x_0/2) \\
& = \frac{1}{\delta} \left( \underbrace{x_0^T (x- x_0)}_{\ge 0} + \underbrace{|x_0|^2/2}_{>0} \right) \\
& > 0
\end{align}
Thus the hyperplane $a^T x + b$ strictly separates $\{ e \}$ and $C$.
\end{proof}

\subsection{Dual convex sets}

As a first application of separating hyperplanes, we discuss next a geometric form of convex duality.

\begin{definition}[Polar dual]
\labdef{polar_dual}
Let $C \subset \R^n$ be closed and convex, with $0 \in C$.
Its polar dual is the set $C^0 = \{ y \in R^n: x^T y \le 1, \forall x \in C \}$.
\end{definition}

\begin{theorem}[Dual convex sets]
\labthm{dual_convex_sets}
Under \refdef{polar_dual}:
\begin{enumerate}
    \item $C^0$ is closed, convex, with $0 \in C^0$.
    \item We have the duality assertion $(C^0)^0 = C$.
\end{enumerate}
\end{theorem}

\begin{proof}
Step 1:
To see $C^0$ is closed, we can equivalently show that $(C^0)^c = \{ y \in \R^n : x^T y > 1, \exists x \in C \}$ is open.
To check openness, we must show that for any $y_0 \in (C^0)^c$, there exists a $r$-ball such that $B_r(y_0) \subset (C^0)^c$.
Let us take a fixed $y_0 \in (C^0)^c$, there exists $x_0 \in C$ with $x_0^T y_0 = \delta > 1$.
Let $r > 0$ satisfy $|x_0| r < \delta - 1$, $y_r \in B_r(y_0)$.
And denote $\Delta y_r = y_r - y_0$.
Clearly, $|\Delta y_r| \le r$ and
\begin{align}
x_0^T y_r &= x_0^T y_0 + x_0^T \Delta y_r \\
& \ge \delta - |x_0^T \Delta y_r| \\
& \ge \delta - |x_0| |\Delta y_r| \\
& \ge \delta - |x_0| r > 1.
\end{align}
So that for all $y_0 \in (C^0)^c$, there exists a $r$-ball such that $B_r(y_0) \subset (C^0)^c$.
Thus $(C^0)^c$ is open and $C^0$ is closed.

Step 2:
$C^0$ is convex.
For any $a, b \in C^0$, for all $x \in C$, we have $x^T a \le 1$ and $x^T b \le 1$.
Consequently, for all $0 \le \theta \le 1$, $x^T (\theta a + (1-\theta) b) \le 1$.
Thus $\theta a + (1-\theta) b \in C^0$ as well.

Step 3:
For any $x \in C$, $x^T 0 = 0 \le 1$.
Therefore $0 \in C^0$.

Step 4:
Note that $(C^0)^0 = \{ z \in \R^n : y^T z \le 1, \forall y \in C^0 \}$.
Let $x \in C$. Then $y^T x \le 1$ for all $y \in C^0$ and thus $x \in (C^0)^0$. 
Consequently $C \subseteq (C^0)^0$.

If $z \in (C^0)^0 - C$, then since $C$ is closed, \refthm{separating_hyperplane} says there exists $a \in \R ^n, b \in \R$ such that 
\begin{enumerate}
    \item $a^T z + b < 0$,
    \item $a^T x + b > 0$ for all $x \in C$.
\end{enumerate}
Since $0 \in C$, 2 implies $b > 0$.
So if we write $y = -\frac{a}{b}$, 2 says also that $y^T x < 1$ for all $x \in C$.
Hence $y \in C^0$.
Since $z \in (C^0)^0$, it follows that $y^T z \le 1$; and therefore $a^T z + b \ge 0$.
But this contradicts 1.
Thus $(C^0)^0-C$ is empty and hence $(C^0)^0 = C$.
\end{proof}

\subsection{Farkas alternative}

Our next goal is the Farkas alternative, a statement about solving vector inequalities.
This turns out to have a surprising geometric interpretation involving separating hyperplanes for certain convex cones.

\begin{definition}[Finite cone]
\labdef{finite_cone}
Let $\{ a^1, \dots, a^n \} \subset \R^m$.
The finite cone generated by $\{ a^1, \dots, a^n \}$ is the set $C = \{ \sum _{i=1}^{n} x_i a^i : x_i \ge 0 \}$.
\end{definition}

Observe that $b \in C$ precisely when we can solve $Ax =b, x \ge 0$, when $A = [a^1 \dots a^n]$ is the $m \times n$ matrix whose columns are $\{ a^1, \dots, a^n \}$.

\begin{definition}[Basic cone]
\labdef{basic_cone}
If $\{ a^1, \dots, a^k \}$ are independent, we call the finite cone they generate a basic cone.
\end{definition}

\begin{lemma}
Suppose $\{ a^1, \dots, a^n \}$ generate the finite cone $C$. 
Let $C_1, \dots, C_q$ be the basic cones generated by all linearly independent subsets of $\{ a^1, \dots, a^n \}$.
Then 
\begin{align}
C = \bigcup _{i=1} ^{q} C_i.
\end{align}
\end{lemma}

\begin{proof}
Obviously, $C_i \subseteq C$, so that $\cup_{i=1}^{q} C_i \subseteq C$.

Now select $b \in C = \{ A x : x \ge 0 \}$. 
There exists a solution of $A x = b, x \ge 0$.
And according to \refthm{LP_basic_solution}, there in fact exists a basic solution $x^*$ such that $A x^* = b, x^* \ge 0$.
This means that the columns $\{a^{j_1}, \dots, a^{j_m} \}$ of $A$ corresponding to the nonzero entries of $x^*$ are independent. 
SO $b$ belongs to the basic cone generated by $\{a^{j_1}, \dots, a^{j_m} \}$ and thus $b \in \cup_{i=1}^{q} C_i$.
This is true for all $b \in C$, consequently $C \subseteq \cup_{i=1}^{q} C_i$.
\end{proof}

\begin{theorem}
\labthm{finite_cone_convex_closed}
Let $C$ be the finite cone generated by $\{ a^1, \dots, a^n \} \subset \R^m$.
Then $C$ is convex and closed.
\end{theorem}

\begin{proof}
Step 1:
We need to show that $C$ is convex.
Let $b^1, b^2 \in C, 0 \le \theta \le 1$.
Then there exist $x^1, x^2$ such that $b^1 = A x^1, x^1 \ge 0$ and $b^2 = A x^2, x^2 \ge 0$.
Therefore $(1-\theta)b^1 + \theta b^2 = A \left( (1-\theta) x^1 + \theta x^2 \right)$ for $(1-\theta) x^1 + \theta x^2 \ge 0$.
Thus $(1-\theta)b^1 + \theta b^2 \in C$, and so $C$ is convex.

Step 2:
Let $C_i$ be a basic cone generated by an independent set$\{ a^{j_1}, \dots, a^{j_l} \} \subseteq \{ a^1, \dots, a^n \}$.
Assume $\{ b^k \}_{k=1}^{\infty} \subset C_i$, with $\lim _{k \to \infty} b^k = b^0$.
We claim that $b^0 \in C_i$ and this will show that $C_i$ is closed.

First, let us write $B = \{ j_1, \dots, j_l \}$.
Since the vectors $\{ a^j : j \in B \}$ are independent, if $u = [u_{j_1} \dots u_{j_l}]^T \in \R^l$ and $\sum _{j \in B} u_j a^j = 0$, it follows that $u = 0$.
Therefore for all $u \in \R^l$ with $|u| = 1, \sum _{j \in B} u_j a^j \neq 0$.
Hence the Extreme Value Theorem implies that there exists $\epsilon > 0$ such that $\min \left\{ |\sum _{j \in B} u_j a^j| : |u| = 1 \right\} = \epsilon > 0$.
Thus if $v \in \R^l$, 
\begin{align}
|\sum _{j \in B} v_j a^j| \ge \epsilon |v|. \labeq{proof_convex_closed_eq_1}
\end{align}
We now turn to the proof of $b^0 \in C_i$.
Observe that we can write $b^k = A x^k$, where $x^k \ge 0, x^k = [0 \ x_{j_1}^k \ 0 \dots x_{j_l}^k \ 0]^T$.
Then $b^k = \sum _{j \in B} x_j^k a^j$, and therefore \arefeq{proof_convex_closed_eq_1} implies 
\begin{align}
|x^k| \le \frac{1}{\epsilon} |b^k|
\end{align}
for $k=1, \dots$
The sequence $\{ x^k \}_{k=1}^{\infty}$ is therefore bounded, and so we can apply the Bolzano-Weierstrass Theorem to extract a convergent subsequence:
\begin{align}
\lim _{j \to \infty} x^{k_j} = x^0.
\end{align}
Then $x^0 \ge 0$ and $A x^0 = \lim _{j \to \infty} A x^{k_j} = \lim _{j \to \infty} b^{k_j} = b^0$.
Furthermore $x_j^0 = 0$ except possibly for the indices $j \in B$.
Hence $b^0 \in C_i$.

Step 3:
So each basic cone $C_i$ is closed.
The finite union of closed sets is closed, and hence $C = \cup _{i=1}^{q} C_i$ is closed.
\end{proof}

In view of the previous theorem, we can apply the Separating Hyperplane
Theorem to a finite cone. 
This has the following major payoff:

\begin{theorem}[Farkas alternative]
\labthm{farkas_alternative}
Let $A$ be an $m \times n$ matrix, $b \in \R^m$.
Then either 
\begin{enumerate}
    \item $A x = b, x \ge 0$ has a solution $x \in \R^n$; or
    \item $A^T y \ge 0, b^T y < 0$ has a solution $y \in \R^m$,
\end{enumerate}
but not both.
\end{theorem}

\begin{proof}
Step 1:
Assume $x$ solves 1, y solves 2.
Then $0 \le x^T (A^T y) = (Ax)^T y = b^T y < 0$ which is a contradiction.
So 1 and 2 cannot both be true.

Step 2:
Suppose 1 fails.
We will show that then 2 must hold.
Now the failure of 1 means $b \in C = \{ Ax : x \ge 0\}$.
Since $C$ is a finite cone, \refthm{finite_cone_convex_closed} tells us  that $C$ is closed and convex.
Then the Separating Hyperplane Theorem (\refthm{separating_hyperplane}) asserts that there exist $a \in \R^m, c \in \R$ such that 
\begin{align}
a^T z + c > 0 \quad (z \in C) \labeq{proof_farkas_alternative_eq_1}
\end{align}
and 
\begin{align}
a^T b + c < 0. \labeq{proof_farkas_alternative_eq_2}
\end{align}
Let $x \ge 0, \mu \ge 0$.
Set $z = \mu A x = A (\mu x) \in C$.
Then according to \arefeq{proof_farkas_alternative_eq_1}, $a^T (\mu A x) + C > 0$.
Dividing by $\mu > 0$ and letting $\mu \to \infty$, we see that $a^T Ax \ge 0$.
So $(A^T a)^T x \ge 0$ for all $x \ge 0$.
Thus $A^T a \ge 0$.
Let $y = a$, then $A^T y \ge 0$.
Put $z = 0$ in \arefeq{proof_farkas_alternative_eq_1}, to deduce that $c > 0$.
Then \arefeq{proof_farkas_alternative_eq_2} says
\begin{align}
b^T y = b^T a < -c < 0.
\end{align}
\end{proof}

\section{Convex functions}

A convex function is a real-valued function such that the region above its graph is a convex set.
Convex functions therefore inherit many useful properties from convex sets.

\subsection{Convex functions of one variable}

\begin{definition}
A function $f: \R \to \R$ is called convex if 
\begin{align}
f(\theta x_1 + (1-\theta) x_2) \le \theta f(x_1) + (1-\theta) f(x_2) \tag{$C_1$} \labeq{C1}
\end{align}
for all $x_1, x_2 \in \R, 0 \le \theta \le 1$.

A function $g: \R \to \R$ is called concave if $-g$ is convex.
\end{definition}

If $f$ is convex, then for all points $x_1, x_2$, the graph of $f$ lies below the line segment  connecting $[x_1 \ f(x)1)]^T$ and
$[x_2 \ f(x_2)]^T$.

It is easy to see that $f:\R \to \R$ is a convex function if and only if its epigraph
\begin{align}
E = \left\{ 
\begin{bmatrix}
x \\
y \\
\end{bmatrix} :
y \ge f(x), x \in \R
\right\} \subset \R^2
\end{align}
is a convex set.

\begin{remark}
It follows by induction that if $f: \R \to \R$ is convex, then
\begin{align}
f (\sum_{i} \theta_i x_i) \le \sum_{i} \theta_i f(x_i)
\end{align}
for all $x_i \in \R$, $\theta_i \ge 0$ and $\sum_{i} \theta_i = 1$.
\end{remark}

\begin{theorem}[Equivalent characterizations of convexity]
If $f:\R \to \R$ is continuously differentiable, then $f$ is convex if and only if 
\begin{align}
f(x_1) + f'(x_1) (x_2 - x_1) \le f(x_2) \tag{$C_2$} \labeq{C2}
\end{align}
for all $x_1, x_2 \in \R$.

If $f$ is twice continuously differentiable, then $f$ is convex if and only if
\begin{align}
f''(x) \ge 0 \tag{$C_3$} \labeq{C3}
\end{align}
for all $x \in \R$.
\end{theorem}

\begin{proof}
Step 1:
Assume $f$ is continuously differentiable, and let us show \arefeq{C1} holds if and only if \arefeq{C2} holds.
So, suppose \arefeq{C1}.
For $\theta > 0$, 
\begin{align}
f(\theta x_2 + (1-\theta) x_1) \le \theta f(x_2) + (1-\theta) f(x_1) \\
\frac{f(x_1 + \theta (x_2 - x_1)) - f(x_1)}{\theta} \le f(x_2) - f(x_1)
\end{align}
Let $\theta \to 0$, to deduce that $f'(x_1) (x_2 - x_1) \le f(x_2) - f(x_1)$ which is \arefeq{C1}.

Now assume \arefeq{C2}.
Then if $w = \theta x_1 + (1-\theta) x_2$, we have
\begin{align}
f(x_1) & \ge f(w) + f'(w) (x_1-w) \\
f(x_2) & \ge f(w) + f'(w) (x_2-w).
\end{align}
So
\begin{align}
\theta f(x_1) + (1-\theta) f(x_2) & \ge f(w) + f'(w) \underbrace{( \theta (x_1 - w) + (1-\theta) (x_2 - w) )}_{=0} \\
& = f(\theta x_1 + (1-\theta) x_2 ).
\end{align}
So we get \arefeq{C1}.

Step 2:

\end{proof}

The condition \arefeq{C3} is especially convenient for checking if a given function is convex or not.
But the graphs of convex functions can have corners, and so convex functions need not be twice or even once continuously differentiable.
However, they are always continuous.

\begin{theorem}[Convex functions are continuous]
If $f:\R \to \R$ is convex, then $f$ is continuous.
\end{theorem}

\begin{proof}
Step 1:
First we show that $f$ is bounded on each interval $[a, b]$ of finite length.
For each $a \le x \le b$, we can write $x = \theta a + (1-\theta) b$ where $0 \le \theta \le 1$.
Therefore,
\begin{align}
f(x) \le \theta f(a) + (1-\theta) f(b) \le \max \ \{ 
|f(a)|, |f(b)| \}.
\end{align}
Now if $\frac{a+b}{2} \le x \le b$, then $\frac{a+b}{2} = \theta x + (1-\theta) a$ for $\theta = \frac{b-a}{2(x-a)}$.
Then $1/2 \le \theta \le 1$ and convexity implies $f(\frac{a+b}{2}) \le \theta f(x) + (1-\theta) f(a)$.
Hence
\begin{align}
f(x) \ge \frac{1}{\theta} \left( f(\frac{a+b}{2}) - (1-\theta) f(a) \right) \ge -2 \left( |f(\frac{a+b}{2})| + |f(a)| \right)
\end{align}
since $\frac{1}{\theta} \le 2$.
Likewise, if $a \le x \le \frac{a+b}{2}$, we also have
\begin{align}
f(x) \ge -2 \left( |f(\frac{a+b}{2})| + |f(a)| \right).
\end{align}
Therefore
\begin{align}
\sup_{[a, b]} |f(x)| \le 4 \max \ \{ |f(a)|, |f(\frac{a+b}{2})|, |f(b)| \} \le \infty.
\end{align}

Step 2:
Then we prove the continuity for $f$ on $[0, 1]$.
Let $0 \le x < y \le 1$.
Then $y = \theta x + (1-\theta) 2$ for $\theta = \frac{2-y}{2-x}$.
Thus $f(y) \le \theta f(x) + (1-\theta) f(2)$, and so
\begin{align}
f(y) - f(x) & \le (1-\theta)(f(2) -f(x)) \\
& = \underbrace{\frac{1}{2-x}}_{\in [1/2, 1]} (y-x) \underbrace{(f(2) - f(x))}_{\le 2 \sup_{[0, 2]} |f|} \\
& \le 2 |y-x| \sup _{[0, 2]} |f(x)|
\end{align}
Similarly, we have $x = \theta y + (1-\theta) (-2)$ for $\theta = \frac{y+2}{x+2}$.
Then
\begin{align}
f(x) - f(y) \le 2 |y-x| \sup _{[-2, 1]} |f(x)|.
\end{align}
Together we have
\begin{align}
|f(y) - f(x)| \le 2 |y-x| \sup _{[-2, 2]} |f(x)|.
\end{align}
Since $\sup _{[-2, 2]} |f(x)|$ is bounded, this inequality implies $f$ is continuous on $[0, 1]$.

Step 3:
We will finish by shifting $f$ on any finite interval $[a, b]$ to $\hat{f}$ that defined on $[0, 1]$.
Let $\hat{f}(x) = f(\frac{x-a}{b-a})$.
The continuity for $\hat{f}$ is also applied for $f$.\sidenote{
If $f$ is continuous, then $\hat{f}(x) = f(ax + b)$ is also continuous.
For all $\epsilon > 0$, there exists a $r_0 > 0$ such that with given $x_0$ for all $y \in B_{r_0}(x_0)$, $|f(y) - f(x_0)| < \epsilon$.
So if we have $x_0 = a x + b$ and $r = r_0 / a$, then within $y \in B_r(x)$, 
\begin{align}
|\hat{f}(y) -\hat{f}(x)| = |f(a y+b) - f(a x+b)| < \epsilon.
\end{align}
So that $\hat{f}$ is continuous.
}
\end{proof}

\subsection{Convex functions of more variables}

\begin{definition}
A function $f: \R^n \to \R$ is called convex if 
\begin{align}
f(\theta x_1 + (1-\theta) x_2) \le \theta f(x_1) + (1-\theta) f(x_2) \tag{$C_1'$} \labeq{C1p}
\end{align}
for all $x_1, x_2 \in \R, 0 \le \theta \le 1$.

A function $g: \R \to \R$ is called concave if $-g$ is convex.
\end{definition}

If $f$ is convex, then for all points $x_1, x_2$, the graph of $f$ lies below the line segment  connecting $[x_1 \ f(x)1)]^T$ and
$[x_2 \ f(x_2)]^T$.

It is easy to see that $f:\R^n \to \R$ is a convex function if and only if its epigraph
\begin{align}
E = \left\{ 
\begin{bmatrix}
x \\
y \\
\end{bmatrix} :
y \ge f(x), x \in \R^n
\right\} \subset \R^{n+1}
\end{align}
is a convex set.

\begin{remark}
It follows by induction that if $f: \R^n \to \R$ is convex, then
\begin{align}
f (\sum_{i} \theta_i x_i) \le \sum_{i} \theta_i f(x_i)
\end{align}
for all $x_i \in \R^n$, $\theta_i \ge 0$ and $\sum_{i} \theta_i = 1$.
\end{remark}

\begin{theorem}[Equivalent characterizations of multivariable convexity]
If $f:\R^n \to \R$ is continuously differentiable, then $f$ is convex if and only if 
\begin{align}
f(x_1) + \grad f(x_1)^T (x_2 - x_1) \le f(x_2) \tag{$C_2'$} \labeq{C2p}
\end{align}
for all $x_1, x_2 \in \R^n$.

If $f$ is twice continuously differentiable, then $f$ is convex if and only if
\begin{align}
\grad ^2 f(x) \simeq 0 \tag{$C_3'$} \labeq{C3p}
\end{align}
for all $x \in \R^n$.
\end{theorem}

\begin{theorem}[Multivariable convex functions are continuous]
If $f:\R^n \to \R$ is convex, then $f$ is continuous.
\end{theorem}

\subsection{Subdifferentials}

\begin{definition}[Subdifferential]
Let $f:\R^n \to \R$ be convex.
For each $x_1 \in \R^n$, we define
\begin{align}
\pp f(x_1) = \{ r \in \R^n : f(x_1) + r^T (x_2 - x_1) \le f(x_2), \forall x_2 \in \R^n \}.
\end{align}
This set is called the subdifferential of $f$ at $x_1$.
\end{definition}

\begin{example}
Let $n=1$ and $f(x) = |x|$.
Then
\begin{align}
\pp f(x) = 
\begin{cases}
\{ -1 \}, & x < 0 \\
[-1, 1], & x = 0 \\
\{ 1 \}, & x > 0 \\
\end{cases}
\end{align}
\end{example}

\begin{theorem}[Properties of subdifferential]
\labthm{properties_subdifferential}
Let $f:\R^n \to \R$ be convex.
Then for each $x \in \R^n$, $\pp f(x)$ is a closed, convex, and non-empty set.
\end{theorem}

\begin{proof}
Step 1:
Convexity.
For any $r_1, r_2 \in \pp f(x)$ and $0 \le \theta \le 1$, 
\begin{align}
f(x) + \left[ \theta r_1 + (1-\theta) r_2 \right]^T (x_1 - x) 
&= \theta \left[ f(x) + r_1 (x_1 - x) \right] + (1-\theta) \left[ f(x) + r_2 (x_1 - x) \right] \\
&\le f(x_1)
\end{align}
Thus, $\left[ \theta r_1 + (1-\theta) r_2 \right] \in \pp f(x)$.

Step 2:
Closeness. 
Assume now $\{r_k\}_{k=1}^{\infty} \subset \pp f(x)$ and $r_0 = \lim_{k \to \infty} r_k$.
Then for each $k$ and each $x_1$, $f(x) + r_k^T (x_1 - x) \le f(x_1)$.
Let $k \to \infty$ to deduce that $f(x) + r_0^T (x_1 - x) \le f(x_1)$ for each $x_1 \in \R^n$ and hence $r_0 \in \pp f(x)$.
Consequently, $\pp f(x)$ is closed.

Step 3:
Non-emptiness.
Select any point $x \in \R^n$. 
Will show that $\pp f(x) \neq \emptyset$.
TODO.

\end{proof}

\subsection{Dual convex functions}

For this section, assume $f:\R^n \to \R$ is convex, with
\begin{align}
\lim _{|x| \to \infty} \frac{f(x)}{|x|} = + \infty.
\end{align}
This is called a super-linear growth condition.

\begin{definition}
For $y \in \R^n$, the dual convex function (or Legendre transform) of $f$ is
\begin{align}
f^{*}(y) = \max _{x \in \R^n} \{ x^T y - f(x) \}.
\end{align}
\end{definition}

\begin{example}
Let $f(x) = \frac{x^2}{2}$ for $x \in \R$.
Then 
\begin{align}
f^{*}(y) = \max _{x \in \R} \{ x y - \frac{x^2}{2} \} = \frac{y^2}{2}
\end{align}
\end{example}

\begin{lemma}[Fenchel-Young inequality]
\lablemma{F-Y inequality}
For all $x, y \in \R^n$ we have
\begin{align}
x^T y \le f(x) + f^{*}(y).
\end{align}
\end{lemma}

\begin{theorem}
The function $f^{*}: \R^n \to \R$ is convex.
\end{theorem}

\begin{proof}
\begin{align}
f^{*}(\theta y_1 + (1-\theta) y_2 ) 
&= \max _{x} \left\{ x^T \left( \theta y_1 + (1-\theta) y_2 \right) - f(x) \right\} \\
&= \max _{x} \left\{ \theta \left[ x^T y_1 - f(x) \right] + (1-\theta) \left[ x^T y_2 - f(x) \right] \right\} \\
&\le \theta \max _{x} \left\{ x^T y_1 - f(x) \right\} + (1-\theta) \max _{x} \left\{ x^T y_2 - f(x) \right\} \\
&= \theta f^*(y_1) + (1-\theta) f^*(y_2) 
\end{align}
\end{proof}

\begin{theorem}
$\lim_{|y| \to \infty} \frac{f^{*}(y)}{|y|} = +\infty$.
\end{theorem} 

\begin{proof}
According to \reflemma{F-Y inequality}, $f(x) + f^{*}(y) \ge x^T y $ for all $x, y$. 
Fix $y \neq 0, \mu > 0$ and let $x = \frac{\mu y}{|y|}$.
Then
\begin{align}
f^{*}(y) 
&\ge \left( \frac{\mu y}{|y|} \right)^T y - f(\frac{\mu y}{|y|}) 
&\ge \mu |y| - \max_{x \in B_{\mu}(0)} f(x)
\end{align}
So
\begin{align}
\frac{f^{*}(y)}{|y|} &\ge mu - \frac{1}{|y|} \max_{x \in B_{\mu}(0)} f(x) \\
\lim _{|y| \to \infty} \frac{f^{*}(y)}{|y|} &\ge \mu
\end{align}
for all $\mu > 0$.
\end{proof}

\begin{theorem}
$f^{**} = f$.
\end{theorem}

\begin{proof}
Step 1:
According to \reflemma{F-Y inequality}, $f(x) \ge x^T y - f^{*}(y)$ for all $x, y$.
\begin{align}
f(x) \ge \max_{y} \left\{ x^T y - f^{*}(y) \right\} = f^{**}(x)
\end{align}

Step 2:
\refthm{properties_subdifferential} tells us that $\pp f(x)$ is non-empty.
Select $r \in \pp f(x)$, then $f(z) \ge f(x) + r^T (z- x)$ for all $z \in \R^n$.
Consequently\sidenote{FIRST $=$,WHY???}, 
\begin{align}
r^T x - f(x) = \max_{z} \{ r^T z - f(z) \} = f^*(r)
\end{align}
and so
\begin{align}
f^{**}(x) = \max_{y} \left\{ x^T y - f^{*}(y) \right\} \ge x^T r - f^{*}(r) = f(x)
\end{align}
\end{proof}

\begin{theorem}[Subdifferentials and dual functions]
For all points $x, y \in \R^n$, the following are equivalent.
\begin{enumerate}
    \item $x^T y = f(x) + f^{*}(y)$;
    \item $y \in \pp f(x)$;
    \item $x \in \pp f^{*}(y)$.
\end{enumerate}
\end{theorem}

\begin{proof}
TODO
\end{proof}

\section{Applications}
TODO
\setchapterstyle{kao}
\setchapterpreamble[u]{\margintoc}
\chapter{Nonlinear Optimization}
\labch{nonlinear_optimization}

In this chapter we examine minimization problems with inequality constraints and study when and how Lagrange multipliers can be used to characterize minimizers.

\section{Inequality constraints}

Assume $f, h_1, \dots, h_p : \R^n \to \R$ are continuously differentiable.

As usual, we write
\begin{align}
h = \begin{bmatrix}
h_1 \\
h_2 \\
\vdots \\
h_p \\
\end{bmatrix}
\end{align}
and 
\begin{align}
\grad h = \begin{bmatrix}
(\grad h_1)^T \\
(\grad h_2)^T \\
\vdots \\
(\grad h_p)^T \\
\end{bmatrix}
= \begin{bmatrix}
\frac{\pp h_1}{\pp x_1} & \frac{\pp h_1}{\pp x_2} & \cdots & \frac{\pp h_1}{\pp x_n} \\
\frac{\pp h_2}{\pp x_1} & \frac{\pp h_2}{\pp x_2} & \cdots & \frac{\pp h_2}{\pp x_n} \\
\vdots & \vdots & \ddots & \vdots \\
\frac{\pp h_p}{\pp x_1} & \frac{\pp h_p}{\pp x_2} & \cdots & \frac{\pp h_p}{\pp x_n} \\
\end{bmatrix}.
\end{align}

We study in this section the constrained optimization problem to find $x_0 \in \R^n$ to 
\begin{align}
\min \ & f(x) \tag{NLP} \labeq{NLP} \\
\st & h(x) \le 0 \nonumber
\end{align}

The requirement that $h_j(x) \le 0$ for $j = 1, \dots, p$ are inequality constraints.
The $j$-th constraint is active if $h_j(x) = 0$.
A point $x$ is feasible for \arefeq{NLP} if $h(x) \le 0$.

A basic question is how to characterize $x_0$ solving \arefeq{NLP}.

\subsection{Constraint qualification}

Suppose hereafter $x_0$ solves \arefeq{NLP}.
Our plan is to make a first variation calculation, but for this we need to be careful in designing an appropriate curve of variations staying within the feasible region.

We write $J = \{ j \in \{1, \dots, p\} : h_j(x_0) = 0 \}$.
These are the indices of the active constraints for $x_0$.

We write $o(t)$ to denote any vector function $r(t)$ such that $\lim _ {t \to 0_+} \frac{|r(t)|}{t} = 0$.

\begin{definition}
We say the constraint qualification condition (CQ) holds at $x_0$ holds at $x_0$ if for each vector $y \in \R^n$ satisfying $y^T \grad h_j(x_0 \le 0$ for all $j \in J$, there exists a continuous curve $\{x(t) : 0 \le t < t_0 \}$ for some $t_0 > 0$ such that $h(x(t)) \le 0$ within $0 \le t < t_0$; and $x(t) = x_0 + t y + o(t)$ as $t \to 0_+$.
\end{definition}

\subsection{Karush-Kuhn-Tucker conditions}

\begin{theorem}
Let $x_0$ solve \arefeq{NLP} and suppose the constraint qualification condition holds at $x_0$.
Then there exist a vector of real numbers $\mu_0 = [\mu_0^1, \dots, \mu_0^p]^T \ge 0$ such that 
\begin{align}
\grad f(x_0) + \mu_0^T \grad h(x_0) = 0.
\end{align}
Furthermore, the vector $\mu_0$ satisfies
\begin{align}
\mu_0^T h(x_0) = 0.
\end{align}
\end{theorem}

We interpret $\mu_0^j$ as the Lagrange multiplier for the constraint $h_j(x_0) \le 0$. 
So for our inequality constrained problem \arefeq{NLP}, we are asserting both that Lagrange multipliers exist and that they are nonnegative.

In addition, if $h_j(x_0) < 0$ for some index $j$, that constraint is inactive and so the corresponding Lagrange multiplier $\mu_0^j$ equals zero.
This is a complementary slackness condition.

\begin{proof}
Step 1.
Assume that the vector $y$ satisfies $y^T \grad h_j(x_0) \le 0$ for $j \in J$.
Let $\{ x(t) : 0 \le t < t_0 \}$ be the corresponding curve, whose existence is assured according to CQ.
Write $\phi(t) = f(x(t))$.
Then $\phi(0) = f(x_0) \le f(x(t)) = \phi(t)$ for $0 \le t < t_0$, since $x_0$ solves \arefeq{NLP}.
Thus $phi$ has a minimum at $t = 0$ and hence $\phi'(0) \ge 0$.
Now $\phi'(0) = \grad f(x_0) ^T x'(0) = \grad f(x_0) ^T y$ and therefore $\grad f(x_0) ^T y \ge 0$ for all $y$ satisfying $y^T \grad h_j(x_0) \le 0$ for $j \in J$. 

Step 2.
Recall that \refthm{farkas_alternative} states:
For an $m \times n$ matrix $A$ and $b \in \R^m$, either 
\begin{enumerate}
    \item $A x = b, x \ge 0$ has a solution $x \in \R^n$; or
    \item $A^T y \ge 0, b^T y < 0$ has a solution $y \in \R^m$,
\end{enumerate}
but not both.

We apply this to 
\begin{align}
A &= - \underbrace{[\grad h_{j_1} (x_0) \ \grad h_{j_2} (x_0)\ \cdots \ \grad h_{j_k} (x_0) ]}_{\text{columns}}, \\
b &= \grad f(x_0),
\end{align}
where $J = \{j_1, j_2, \dots, j_k\}$.
In Step 1, we have shown that $A^T y \ge 0$ implies $y^T b \ge 0$ and thus fails.

Consequently, 2 holds: there exists $\sigma_j \ge 0$ for $j \in J$ such that 
\begin{align}
- \sum_{j \in J} \sigma_j \grad h_j(x_0) = \grad f(x_0).
\end{align}
Define $\mu _0 \in \R^p$ by
\begin{align}
\mu_0^j = \begin{cases}
\sigma_j, & j \in J \\
0, & j \notin J \\
\end{cases}
\end{align}
then $\mu_0 \ge 0$, $\mu_0^T h(x_0) = 0$ and $ \grad f(x_0) + \mu_0^T h(x_0) = 0$.
\end{proof}

\subsection{When does the constraint qualification condition hold?}

The above proof is elegant, but it may be far from clear for particular problems if the constraint qualification condition is valid. We discuss next two important cases.

\subsubsection{Linear inequality and equality constraints}

\begin{theorem}
If the functions $\{ h_j \}_{j=1}^{p}$ are linear (or affine) functions of $x$, then (CQ) holds for each point $x_0$.
\end{theorem}

\subsubsection{Regular equality constraints}

\begin{definition}
We say that $x_0$ is regular for \arefeq{NLP} if $\{ \grad h_j(x_0) \}_{j \in J}$ are linearly independent in $\R^n$ where $J = \{ j \in \{1, \dots, p\} : h_j(x_0) = 0 \}$.
\end{definition}

\begin{theorem}
If $x_0$ is regular for \arefeq{NLP} then (CQ) holds at $x_0$.
\end{theorem}

\section{More on Lagrange multipliers}

Given $f, g_1, \dots, g_m, h_1, \dots, h_p : \R^n \to \R$, we write 
\begin{align}
g = \begin{bmatrix}
g_1 \\
g_2 \\
\vdots \\
g_m \\
\end{bmatrix} \\
h = \begin{bmatrix}
h_1 \\
h_2 \\
\vdots \\
h_p \\
\end{bmatrix}.
\end{align}
Our minimization problem for this section is to find $x_0 \in \R^n$ to 
\begin{align}
\min \ & f(x) \tag{NLP*} \labeq{NLPs} \\
\st    & g(x) = 0 \nonumber \\
       & h(x) \le 0 \nonumber
\end{align}
So there are $m$ equality constraints and $p$ inequality constraints.

\subsection{F. John's formulation}

\begin{theorem}
Suppose that $x_0$ solves the constrained optimization problem \arefeq{NLPs}.
Then there exists real number $\gamma_0$ and real-valued vectors $\lambda_0, \mu_0$ not all equal to zero such that 
\begin{align}
\gamma_0 \grad f(x_0) + \lambda_0^T \grad g(x_0) + \mu_0^T \grad h(x_0) = 0
\end{align}
and 
\begin{align}
\gamma_0 \ge 0, \mu_0 \ge 0, \mu_0^T h(x_0) = 0.
\end{align}
\end{theorem}

\begin{remark}
For $\gamma_0 > 0$, by dividing $\lambda_0, \mu_0$ by $\gamma_0$, we convert F. John's condition to KKT conditions.
If $\gamma_0 = 0$, we call it an abnormal multiplier.
\end{remark}

\section{Quadratic programming}

The quadratic programming is to find $x_0 \in \R^n$ to 
\begin{align}
\min \ & \frac{1}{2} x^T C x + c^T x \tag{QP} \labeq{QP} \\
\st    & A x = b \nonumber \\
       & x \ge 0 \nonumber
\end{align}
where $C$ is a symmetric $n \times n$ matrix.

The problem \arefeq{QP} is of the form \arefeq{NLPs} for $f(x) = \frac{1}{2} x^T C x + c^T x$, $g = Ax - b$, $h = -x$.


\setchapterstyle{kao}
\setchapterpreamble[u]{\margintoc}
\chapter{Convex Optimization}
\labch{convex_optimization}

We now make additional convexity assumptions, which will let us greatly
strengthen the theory from the previous chapter.

\section{Variational inequalities}

Let $C \subset \R^n$ be a convex set.
We begin with the basic optimization problem of finding $x_0$ to

\begin{align}
    \min f(x), \st x \in C. \tag{C}
\end{align}

\begin{theorem}
    If $f: \R^n \to \R$ is continuously differentiable and $x_0$ solves , then 
    \begin{align}
        \grad f(x_0)^T (x-x_0) \ge 0, \text{ for all } x \in C. \tag{VI}
    \end{align}
    Moreover, suppose also that $f$ is convex.
    Then if $x_0 \in C$ satisfies , $x_0$ solves the minimization problem .
\end{theorem}

\begin{proof}
    Let $x \in C$.
    Then $x_0 + t(x - x_0) = t x + (1-t) x_0 \in C$ for $0 \le t \le 1$.
    Thus if $x_0$ solves , it follows that $\phi(t) = f(x_0 + t(x - x_0))$ has its minimum for $0 \le t \le 1$ at $t = 0$.
    Therefore $\phi'(0) \ge 0$.
    But $\phi'(t) = \grad f(x_0 + t(x - x_0))^T (x-x_0)$ and hence $\phi'(0) = \grad f(x_0)^T (x-x_0) \ge 0$.
    
    If $f$ is convex and continuously differentiable, then $f(x) \ge f(x_0) + \grad f(x_0)^T (x-x_0)$.
    Since $\grad f(x_0)^T (x-x_0) \ge 0$, $f(x) \ge f(x_0)$ for all $x \in C$.
\end{proof}

\section{Convexity and Lagrange multipliers}

\subsection{Sufficient condition for minimality}

\subsection{Slater's condition}

\begin{definition}
    We say that Slater's condition for holds provided
    \begin{align}
        \text{there exists}
    \end{align}
\end{definition}

\pagelayout{wide} % No margins
\addpart{Dynamic Optimization}
\pagelayout{margin} % Restore margins

% \input{chapters/layout.tex}
% \setchapterstyle{kao}
\setchapterpreamble[u]{\margintoc}
\chapter{Mathematics and Boxes}
\labch{mathematics}

\section{Theorems}

Despite most people complain at the sight of a book full of equations, 
mathematics is an important part of many books. Here, we shall 
illustrate some of the possibilities. We believe that theorems, 
definitions, remarks and examples should be emphasised with a shaded 
background; however, the colour should not be to heavy on the eyes, so 
we have chosen a sort of light yellow.\sidenote[][*10]{The boxes are all of the 
same colour here, because we did not want our document to look like 
\href{https://en.wikipedia.org/wiki/Harlequin}{Harlequin}.}

\begin{definition}
\labdef{openset}
Let $(X, d)$ be a metric space. A subset $U \subset X$ is an open set 
if, for any $x \in U$ there exists $r > 0$ such that $B(x, r) \subset 
U$. We call the topology associated to d the set $\tau\textsubscript{d}$ 
of all the open subsets of $(X, d).$
\end{definition}

\refdef{openset} is very important. I am not joking, but I have inserted 
this phrase only to show how to reference definitions. The following 
statement is repeated over and over in different environments.

\begin{theorem}
A finite intersection of open sets of (X, d) is an open set of (X, d), 
i.e $\tau\textsubscript{d}$ is closed under finite intersections. Any 
union of open sets of (X, d) is an open set of (X, d).
\end{theorem}

\begin{proposition}
A finite intersection of open sets of (X, d) is an open set of (X, d), 
i.e $\tau\textsubscript{d}$ is closed under finite intersections. Any 
union of open sets of (X, d) is an open set of (X, d).
\end{proposition}

\marginnote{You can even insert footnotes inside the theorem 
environments; they will be displayed at the bottom of the box.}

\begin{lemma}
A finite intersection\footnote{I'm a footnote} of open sets of (X, d) is 
an open set of (X, d), i.e $\tau\textsubscript{d}$ is closed under 
finite intersections. Any union of open sets of (X, d) is an open set of 
(X, d).
\end{lemma}

You can safely ignore the content of the theorems\ldots I assume that if 
you are interested in having theorems in your book, you already know 
something about the classical way to add them. These example should just 
showcase all the things you can do within this class.

\begin{corollary}[Finite Intersection, Countable Union]
\labcorollary{finite_intersection}
A finite intersection of open sets of (X, d) is an open set of (X, d), 
i.e $\tau\textsubscript{d}$ is closed under finite intersections. Any 
union of open sets of (X, d) is an open set of (X, d).
\end{corollary}

\begin{proof}
The proof is left to the reader as a trivial exercise. Hint: \blindtext
\end{proof}

\begin{definition}
Let $(X, d)$ be a metric space. A subset $U \subset X$ is an open set 
if, for any $x \in U$ there exists $r > 0$ such that $B(x, r) \subset 
U$. We call the topology associated to d the set $\tau\textsubscript{d}$ 
of all the open subsets of $(X, d).$
\end{definition}

\marginnote{
	Here is a random equation, just because we can:
	\begin{equation*}
  x = a_0 + \cfrac{1}{a_1
          + \cfrac{1}{a_2
          + \cfrac{1}{a_3 + \cfrac{1}{a_4} } } }
	\end{equation*}
}

\begin{example}
Let $(X, d)$ be a metric space. A subset $U \subset X$ is an open set 
if, for any $x \in U$ there exists $r > 0$ such that $B(x, r) \subset 
U$. We call the topology associated to d the set $\tau\textsubscript{d}$ 
of all the open subsets of $(X, d).$
\end{example}

\begin{remark}
Let $(X, d)$ be a metric space. A subset $U \subset X$ is an open set 
if, for any $x \in U$ there exists $r > 0$ such that $B(x, r) \subset 
U$. We call the topology associated to d the set $\tau\textsubscript{d}$ 
of all the open subsets of $(X, d).$
\end{remark}

As you may have noticed, definitions, example and remarks have independent counters; theorems, propositions, lemmas and corollaries share the same counter. \sidenote{\refdef{openset}, \refcorollary{finite_intersection}}

\begin{remark}
Here is how an integral looks like inline: $\int_{a}^{b} x^2 dx$, and 
here is the same integral displayed in its own paragraph:
\[\int_{a}^{b} x^2 dx\]
\end{remark}


There is also an environment for exercises.
\begin{align}
\int _{\C} \frac{\pp y}{\pp z}\dd x
\end{align}

\begin{exercise}
Prove (or disprove) the Riemann hypothesis.
\end{exercise}

We provide one package for the theorem styles: 
\href{kaotheorems.sty}{kaotheorems.sty}, to which you can pass the 
\Option{framed} option you do want coloured boxes around theorems, like 
in this document.\sidenote{The styles without \Option{framed} are not 
showed, but actually the only difference is that they don't have the 
yellow boxes.} You may want to edit this files according to your taste 
and the general style of the book. However, there is an option to 
customise the background colour of the boxes if you use the 
\Option{framed} option: when you load this package, you can pass it the 
\Option{background=mycolour} option (replace \enquote{mycolour} with the 
actual colour, for instance, \enquote{red!35!white}). This will change 
the colour of all the boxes, but it is also possible to override the 
default colour only for some elements. For instance, the 
\Option{propositionbackground=mycolour} option will change the colour 
for propositions only. There are similar options for theorem, 
definition, lemma, corollary, remark, and example.

\section[Boxes \& Environments]{Boxes \& Custom Environments}
\sidenote[][*1.8]{Notice that in the table of contents and in the 
	header, the name of this section is \enquote{Boxes \& Environments}; 
	we achieved this with the optional argument of the \texttt{section} 
	command.}

Say you want to insert a special section, an optional content or just 
something you want to emphasise. We think that nothing works better than 
a box in these cases. We used \Package{mdframed} to construct the ones 
shown below. You can create and modify such environments by editing the 
provided file \href{style/environments.sty}{environments.sty}.

\begin{kaobox}[frametitle=Title of the box]
\blindtext
\end{kaobox}

If you set up a counter, you can even create your own numbered 
environment.

\begin{kaocounter}
	Asshole.
\end{kaocounter}

\section{Experiments}

It is possible to wrap marginnotes inside boxes, too. Audacious readers 
are encouraged to try their own experiments and let me know the 
outcomes.

\marginnote[-2.2cm]{
	\begin{kaobox}[frametitle=title of margin note]
		Margin note inside a kaobox.\\
		(Actually, kaobox inside a marginnote!)
	\end{kaobox}
}

I believe that many other special things are possible with the 
\Class{kaobook} class. During its development, I struggled to keep it as 
flexible as possible, so that new features could be added without too 
great an effort. Therefore, I hope that you can find the optimal way to 
express yourselves in writing a book, report or thesis with this class, 
and I am eager to see the outcomes of any experiment that you may try.

% \begin{margintable}
% 	\captionsetup{type=table,position=above}
% 	\begin{kaobox}
% 		\caption{caption}
% 		\begin{tabular}{ |c|c|c|c| }
% 			\hline
% 			col1 & col2 & col3 \\
% 			\hline
% 			\multirow{3}{4em}{Multiple row} & cell2 & cell3 \\ & cell5 
% 			%& cell6 \\ 
% 			& cell8 & cell9 \\
% 			\hline
% 		\end{tabular}
% 	\end{kaobox}
% \end{margintable}


% \appendix % From here onwards, chapters are numbered with letters, as is the appendix convention

% \pagelayout{wide} % No margins
% \addpart{Appendix}
% \pagelayout{margin} % Restore margins

% \input{chapters/appendix.tex}

%----------------------------------------------------------------------------------------

\backmatter % Denotes the end of the main document content
\setchapterstyle{plain} % Output plain chapters from this point onwards

%----------------------------------------------------------------------------------------
%	BIBLIOGRAPHY
%----------------------------------------------------------------------------------------

% The bibliography needs to be compiled with biber using your LaTeX editor, or on the command line with 'biber main' from the template directory

\defbibnote{bibnote}{References are listed in citation order.\par\bigskip} % Prepend this text to the bibliography
\printbibliography[heading=bibintoc, title=Bibliography, prenote=bibnote] % Add the bibliography heading to the ToC, set the title of the bibliography and output the bibliography note

%----------------------------------------------------------------------------------------
%	NOMENCLATURE
%----------------------------------------------------------------------------------------

% The nomenclature needs to be compiled on the command line with 'makeindex main.nlo -s nomencl.ist -o main.nls' from the template directory

% \nomenclature{$c$}{Speed of light in a vacuum inertial frame}
% \nomenclature{$h$}{Planck constant}

% \renewcommand{\nomname}{Notation} % Rename the default 'Nomenclature'
% \renewcommand{\nompreamble}{The next list describes several symbols that will be later used within the body of the document.} % Prepend this text to the nomenclature

% \printnomenclature % Output the nomenclature

%----------------------------------------------------------------------------------------
%	GREEK ALPHABET
% 	Originally from https://gitlab.com/jim.hefferon/linear-algebra
%----------------------------------------------------------------------------------------

% \vspace{1cm}

% {\usekomafont{chapter}Greek Letters with Pronunciations} \\[2ex]
% \begin{center}
% 	\newcommand{\pronounced}[1]{\hspace*{.2em}\small\textit{#1}}
% 	\begin{tabular}{l l @{\hspace*{3em}} l l}
% 		\toprule
% 		Character & Name & Character & Name \\ 
% 		\midrule
% 		$\alpha$ & alpha \pronounced{AL-fuh} & $\nu$ & nu \pronounced{NEW} \\
% 		$\beta$ & beta \pronounced{BAY-tuh} & $\xi$, $\Xi$ & xi \pronounced{KSIGH} \\ 
% 		$\gamma$, $\Gamma$ & gamma \pronounced{GAM-muh} & o & omicron \pronounced{OM-uh-CRON} \\
% 		$\delta$, $\Delta$ & delta \pronounced{DEL-tuh} & $\pi$, $\Pi$ & pi \pronounced{PIE} \\
% 		$\epsilon$ & epsilon \pronounced{EP-suh-lon} & $\rho$ & rho \pronounced{ROW} \\
% 		$\zeta$ & zeta \pronounced{ZAY-tuh} & $\sigma$, $\Sigma$ & sigma \pronounced{SIG-muh} \\
% 		$\eta$ & eta \pronounced{AY-tuh} & $\tau$ & tau \pronounced{TOW (as in cow)} \\
% 		$\theta$, $\Theta$ & theta \pronounced{THAY-tuh} & $\upsilon$, $\Upsilon$ & upsilon \pronounced{OOP-suh-LON} \\
% 		$\iota$ & iota \pronounced{eye-OH-tuh} & $\phi$, $\Phi$ & phi \pronounced{FEE, or FI (as in hi)} \\
% 		$\kappa$ & kappa \pronounced{KAP-uh} & $\chi$ & chi \pronounced{KI (as in hi)} \\
% 		$\lambda$, $\Lambda$ & lambda \pronounced{LAM-duh} & $\psi$, $\Psi$ & psi \pronounced{SIGH, or PSIGH} \\
% 		$\mu$ & mu \pronounced{MEW} & $\omega$, $\Omega$ & omega \pronounced{oh-MAY-guh} \\
% 		\bottomrule
% 	\end{tabular} \\[1.5ex]
% 	Capitals shown are the ones that differ from Roman capitals.
% \end{center}

%----------------------------------------------------------------------------------------
%	GLOSSARY
%----------------------------------------------------------------------------------------

% The glossary needs to be compiled on the command line with 'makeglossaries main' from the template directory

\setglossarystyle{listgroup} % Set the style of the glossary (see https://en.wikibooks.org/wiki/LaTeX/Glossary for a reference)
\printglossary[title=Special Terms, toctitle=List of Terms] % Output the glossary, 'title' is the chapter heading for the glossary, toctitle is the table of contents heading

%----------------------------------------------------------------------------------------
%	INDEX
%----------------------------------------------------------------------------------------

% The index needs to be compiled on the command line with 'makeindex main' from the template directory

\printindex % Output the index

%----------------------------------------------------------------------------------------
%	BACK COVER
%----------------------------------------------------------------------------------------

% If you have a PDF/image file that you want to use as a back cover, uncomment the following lines

%\clearpage
%\thispagestyle{empty}
%\null%
%\clearpage
%\includepdf{cover-back.pdf}

%----------------------------------------------------------------------------------------

\end{document}
